\documentclass[titlepage,twocolumn,a4paper,10pt,fleqn,leqno]{article}

\usepackage{hyperref}
\usepackage[utf8]{inputenc}
\usepackage[T1]{fontenc}
\usepackage{polski}
\usepackage{titlesec}
\usepackage{wrapfig}
\usepackage{listings}
\usepackage{blindtext}
\usepackage{xcolor}

\definecolor{codegreen}{rgb}{0,0.6,0}
\definecolor{codegray}{rgb}{0.5,0.5,0.5}
\definecolor{codepurple}{rgb}{0.58,0,0.82}
\definecolor{backcolour}{rgb}{0.95,0.95,0.95}

\lstdefinestyle{cpp-style}{
    backgroundcolor=\color{backcolour},
    commentstyle=\color{codegreen},
    keywordstyle=\color{magenta},
    numberstyle=\tiny\color{codegray},
    stringstyle=\color{codepurple},
    basicstyle=\ttfamily\scriptsize,
    breakatwhitespace=false,
    breaklines=true,
    captionpos=b,
    keepspaces=false,
    numbers=left,
    numbersep=1pt,
    showspaces=false,
    showstringspaces=false,
    showtabs=false,
    tabsize=2
}

\newcommand\includeCpp[2]{\noindent\lstinputlisting[language=C++,style=cpp-style,captionpos=t,title=\scriptsize#2]{#1/#2}}
\newcommand\includeText[2]{\noindent\lstinputlisting[captionpos=t,style=cpp-style,title=\scriptsize#2]{#1/#2}}

\newcommand\inlineBash[1]{[\nobreak\lstinline[language=bash,style=cpp-style,frame=lines]{$> #1}\nobreak]}

\newcommand\inlineCpp[1]{[\nobreak\lstinline[language=C++,style=cpp-style]{#1}\nobreak]}

\titleformat{\section}
  {\normalfont\large\bfseries}{\thesection.}{1em}{}
\titleformat{\subsection}
  {\normalfont\normalsize\itshape}{\thesubsection.}{1em}{}
\titleformat{\subsubsection}
            {\normalfont\normalsize\itshape}{\thesubsubsection.}{1em}{}

\usepackage{relsize,endnotes,xurl,hyperref}

\renewcommand\UrlFont{\rmfamily}

\newcommand{\note}[2]{#1\endnote{#2}}

\newcommand{\link}[1]{\url{#1}}

\newcommand{\mail}[1]{\href{mailto:#1}{#1}}

\newcommand{\Rsign}[1]{\protect\hspace{-.15em}\protect\raisebox{.4ex}{\smaller\smaller\smaller\textbf{#1}}}
\newcommand{\Cpp}[1]{\mbox{C\Rsign{+}\Rsign{+}\protect\hspace{-.15em}#1}}

\makeatletter
\renewcommand\paragraph{%
  \@startsection{paragraph}{4}{0mm}%
                {-\baselineskip}%
                {0.5\baselineskip plus 0.2\baselineskip minus 0.2\baselineskip}%
                {\normalfont\normalsize\bfseries}}
\makeatother

\newcommand\editNote[2]{#1\textit{\color{red}{<<#2>>}}}

\newcommand\code[1]{\texttt{#1}}
\newcommand\cname[1]{\code{#1}}
\newcommand\fname[1]{\code{#1()}}
\newcommand\tname[1]{\code{#1<>}}

\usepackage{multirow,graphicx}

\newcommand\setTitle[1]{\title{#1}\def\titleText{#1}}
\newcommand\setTitlePhoto[1]{\def\titlePhoto{#1}}
\newcommand\setTitleDescription[1]{\def\titleDescription{#1}}
\newcommand\setTitleCatchphrases[1]{\def\titleCatchphrases{#1}}
\newcommand\setAuthor[1]{\author{#1}\def\authorName{#1}}
\newcommand\authorPhoto[1]{\def\authorPhoto{#1}}
\newcommand\authorPosition[1]{\def\authorPosition{#1}}
\newcommand\authorWebpage[1]{\def\authorWebpage{#1}}
\newcommand\authorEmail[1]{\def\authorEmail{#1}}

\newcommand\getAuthor{\authorName}
\newcommand\getTitle{\titleText}
\newcommand\getTitleCatchphrases{\titleCatchphrases}

\newcommand\printAuthor{%
\setlength\tabcolsep{0pt}%
\begin{tabular}{p{.89\textwidth}p{.10\textwidth}p{.01\textwidth}}
\Large\getAuthor & \multirow{4}{*}{\includegraphics[height=4.1\baselineskip]{\authorPhoto}} &\\
\footnotesize\authorPosition & &\\
\footnotesize\authorWebpage & & \\
\footnotesize\mail{\authorEmail} & &\\
\end{tabular}
}
\newcommand\printTitle{\center{\mbox{\Huge\bf{\getTitle}}}}
\newcommand\printCatchphrases{\footnotesize{\getTitleCatchphrases}}
\newcommand\printTitlePhoto{\includegraphics[width=.8\textwidth]{\titlePhoto}}
\newcommand\printTitleDescription{\framebox[\linewidth]{%
\parbox{0.9\linewidth}{\vspace{10pt}\textit{\titleDescription}\vspace{10pt}}%
}}

\renewcommand\maketitle{\noindent
\printAuthor
\rule{\textwidth}{1pt}
\vspace{2cm}
\printTitle\\
\vspace{1cm}
\center{\printTitlePhoto}\\
\vspace{1cm}
\center{\printTitleDescription}\\
\center{\printCatchphrases}
}


\setTitle{Współdzielenie zasobów}
\setTitleCatchphrases{\Cpp, Przetwarzanie współbieżne, Struktura programu}
\setTitlePhoto{images/pizza}
\setTitleDescription{Artykuł prezentuje wybrane metody dostępu do zasobów w przykładach wzorowanych na kodzie produkcyjnym. Autor omawia problematykę, a następnie wskazuje metody usystematyzowania dostępu do elementów współdzielonych.}

\setAuthor {Bartosz Charuza}
\authorPhoto {images/author}
\authorPosition {Regular Software Developer}
\authorProject {Alstom: Ebiscreen2000}
\authorWebpage {\link{https://github.com/bcharuza}}
\authorEmail {bcharuza@yahoo.com}
\authorDesc {Programista C++ specjalizujący się w wysokopoziomowej komunikacji między usługami sieciowymi. 5 letnie doświadczenie w sterownikach przemysłowych. Od roku pracuje nad systemami konroli kolei.}

\hypersetup{pdfauthor=\getAuthor,pdftitle=\getTitle,pdfpagemode=None,colorlinks=true,linkcolor=black,urlcolor=black}


\begin{document}
\begin{titlepage}
\maketitle
\end{titlepage}
\section{Wstęp}\label{sec:introduction}
\paragraph{Dlaczego napisałem ten artykuł?}
Historia tego artykułu miała początek pewnego pięknego październikowego ranka, krótko po załączeniu komputera otrzymałem radosne przywitanie. ,,Hello. How are you?''.\\
Poniżej czaiły się 3 kropki grozy\\
...\\
\\
,,Sometimes X doesn't work'' - oto mój bilet na autystyczne randez-vous z tysiącami linii kodu.
,,Sometimes'' jest szczególnie złowieszczym słowem, gdyż ujawnia wyścig - teraz będę  godzinami gonił czerwoną kropkę drwiącego losu. Po 2 dniach galerniczej przygody odkryłem skarb! Wyścig był powodowany zajęciem niewłaściwego muteksu. Nastąpił przełom - \note{niczym oporna drukarka, która zmieniła losy świata}{\link{https://en.wikipedia.org/wiki/Richard\_Stallman\#Events\_leading\_to\_GNU}}, ten niedziałający muteks zmobilizował mnie do przepisania biblioteki i napisania serii artykułów.

\paragraph{Kto jest adresatem?}
Programiści \Cpp{}, od juniora który może znaleźć inspirację, po seniora który może chcieć wymagać podobnych rozwiązań w swoim projekcie. Zakładam że czytelnik posiada podstawową znajomość \Cpp{14} i wiedzę z zakresu przetwarzania wielowątkowego z użyciem mutexów biblioteki standardowej.

\paragraph{Jaki jest zakres artykułu?}
Publikacja prezentuje wzorzec \tname{Resource} i przypadki użycia. Omawiane będą kontrprzykłady, zagadnienia inicjacji i dostępu do zasobów. Nie opisuję narzędzi, ani podstaw wielowątkowości. Większość poruszanych tematów będzie związana z wykorzystaniem wspomnianego wzorcu.

Prezentowane metody sprawdzone są w aplikacjach klasy desktop przy użyciu mechanizmów biblioteki standardowej do kilkudziesięciu wątków. Nie będą poruszane rozwiązania dla aplikacji serwerowych, czy o wysokich wymaganiach wydajnościowych w których nie mam doświadczenia. Artykuł nie ma też zastosowania dla problematyki massive-parallelizm.

\paragraph{Struktura artykułu}
Próbowałem napisać ten artykuł w standardowej akademickiej strukturze problem-analiza-rozwiązanie. Skutkowało to zbyt rozciągłą formą wraz dojściem nowych wątków, dywagacji i przypadków szczególnych.

Zdecydowałem się zastosować formułę rozwiązanie-zastosowanie, gdzie przypadki użycia są podzielone na krótkie pary kontrprzykład-przykład. Mam nadzieję że pozwoli to zredukować rozmiar do minimum i dotrwać czytelnikowi do końca artykułu.

\section{Przygotowanie merytoryczne}\label{sec:reader-profile}
\paragraph{Oznaczenia}
\begin{itemize}
\item \cname{Nazwy\_systemowe}
\item \fname{Nazwy\_funkcji}
\item \tname{Nazwy\_szablonów}
\end{itemize}
\paragraph{Zasób}
\note{Zasób to coś o ograniczonej dostępności.}{\link{https://en.wikipedia.org/wiki/System\_resource}} Zasoby w przeciwieństwie do informacji/wartości posiadają tożsamość. Zasobem może być urządzenie zewnętrzne, pamięć, czas procesora, a nawet obiekt.

\paragraph{Typ}
\note{Typ określa zestaw dozwolonych wartości i operacji obiektu.}{Bjarne Stroustrup: ,,Język \Cpp. Kompendium wiedzy.''}

\paragraph{Wartość}
\note{Wartość to zbiór bitów interpretowanych zgodnie z typem}{Bjarne Stroustrup: ,,Język \Cpp. Kompendium wiedzy.''}

\paragraph{Obiekt}
\note{Obiekt to miejsce w pamięci, w którym przechowywana jest wartość jakiegoś typu}{Bjarne Stroustrup: ,,Język \Cpp. Kompendium wiedzy.''}

\paragraph{Sekcja krytyczna}
\note{Sekcją krytyczną jest zestaw instrukcji chroniony przed jednoczesnym wywołaniem względem danego zasobu}{\link{https://en.wikipedia.org/wiki/Critical\_section}}

\paragraph{Znajomość \Cpp{14}}
Większość kodu będzie napisana w \Cpp{14}. Czytelnik znający \Cpp{11} nie powinien widzieć różnicy. Czytelnik znający \Cpp{03} powinien być w stanie zrozumieć treść. Starsze standardy różnią się zbyt mocno aby artykuł mógł być łatwo przyswajalny.

\paragraph{Przetwarzanie współbieżne}
Znajomość mutexów i podstaw wielowątkowości biblioteki standardowej \Cpp{} powinna być wystarczająca.

\paragraph{Kompilacja i linkowanie}
By zrozumieć niektóre problemy czytelnik powinien znać proces kompilacji i linkowania plików źródłowych do binarnych.

\paragraph{Programowanie szablonowe}
Część kodu jest kodem szablonowym. Do pełnego zorumienia implementacji należy rozumieć podstawy programowania szablonowego i narzędzia szablonowe biblioteki standardowej.

\section{Środowisko testowe}\label{sec:environment}
\begin{tabular}{l r}
  Kompilator:   & GCC v9.3\\
  System:       & Ubuntu 20.04.2 LTS\\
  Architektura: & x86\_64
\end{tabular}

\section{Opis problemu}\label{sec:problem-desc}
Za punkt wyjścia przyjmiemy spreparowany omówiony poniżej, które można znaleść w katalogu \code{./examples/problem} \note{projektu}{\link{https://github.com/bcharuza/resource-management/}}. Przykłady są syntezą kodu produkcyjnego zastanego w kilku projektach z którymi pracowałem.

\paragraph{main}
Main wskazuje przykładowy kod programu. Dobrym ćwiczeniem dla czytelnika byłoby wskazanie kilku poważnych błędów zwązanych z nieprawidłowym użyciem przedstawionych uprzednio struktur.
\includeCpp{../../examples/problem}{main.cpp}
Podstawowym problemem tego kodu jest to że uprzednie struktury zezwoliły ich użytkownikowi na ich nieprawidłowe użycie. Interface jest nieprawidłowo skonstruowany. Co gorsze po wykonaniu pierwszego testu za pomocą \inlineBash{ctest -R problem-test1} (lub \inlineBash{<problem-binary> test1.cfg} gdy kompilujemy bez CMake) aplikacja się zawiesza.
\includeText{../../tests/}{test1.cfg}
\includeText{../../tests/}{test1-in.txt}
Zawieszenie się aplikacji jest spowodowane podwójnym zajęciem muteksu klasy. Każda z metod zajmuje muteks instancji by zablokować klasę. W przypadku gdy \code{std::mutex} jest zajmowany poraz 2 w tym samym wątku następuje zakleszczenie. Jest to znacznie trudniejsze do wykrycia gdy muteks jest statyczny, lub co gorsze - współdzielony w pamięci wolnej. Można to rozwiązać zamieniając \code{std::mutex} na \code{std::recursive\_mutex} \note{jednak jego użycie uznaje się za symptom źle zaprojektowanego współdzielenia zasobów}{Anthony Williams ,,Język \Cpp{} i przetwarzanie współbieżne w akcji''} - w skrócie jeśli pozwalamy na zakleszczenie na pojedynczym muteksie, prawdopodobnie kod jest też wrażliwy na zakleszczenie w scenariuszach z wieloma muteksami, a sekcje krytyczne są zbyt rozbudowane.

Podwójne blokowanie można naprawić przez zamianę definicji \inlineCpp{using mtx\_t = std::mutex} na \inlineCpp{using mtx\_t = std::recursive\_mutex} w pliku \code{MessageHandler.hpp}. Kod nadal będzie posiadał wyścig wewnątrz \code{ping\_race()}, ale logi początkowe będą możliwe do odczytania. Wyścig w \code{pingrace()} jest spowodowany umieszczeniem muteksu i zasobów w różnych zakresach - muteks jest lokalny dla instancji, zaś zasoby są współdzielone przez obiekty.
Po ponownym uruchomieniu testu otrzymamy następujący wynik:
\begin{itemize}
\item \code{test1-out.txt}: pusty lub wypełniony tylko początkowymi wiadomościami. Nie widać tego w kodzie - jednak jak opisano wcześniej, globalna instancja \code{s\_handler} i lokalna wątku \code{handler1} wpływają na siebie.
\item \code{log}: zawiera wiadomość diagnostyczną i zdublowane wiadomości które miały trafić do \code{test1-out.txt}. Dublowanie wiadomości jest także spowodowane tym że \code{sender} został 2-krotnie zasubskrybowany we współdzielonej przestrzeni pomimo że wygląda jakby był zasubskrybowany dla obiektu.
\item \code{konsola}: Kolejne błędy w kodzie można dostrzec po sprawdzeniu wyjścia konsoli.
  \begin{itemize}
  \item \code{((null)[140624307808064])[INFO] Initialized MsgHandler}: trace został wywołany zanim zainicjowano zmienną statyczną \code{CfgReader::s\_sysName}.
    \item \code{(Sys1[140250092123968])[INFO] msgInput:  śčĄţ} Błędy są spowodowane nieprzekazaniem właściwego typu do trace. Kompilator nie był w stanie tego wychwycić.
  \end{itemize}
\end{itemize}

\code{main.cpp} posiada jeszcze jeden błąd trudniejszy do uchwycenia - wyścig instrukcji podczas usuwania elementów kolejki - pomiędzy wywołaniem \code{top()} i \code{remove()} inny proces może też wykonać \code{set()} a wtedy \code{remove()} spowoduje usunięcie wiadomości niebędącej już na szczycie. Jest to spowodowane otwarciem nieformalnej traksakcji -- założeniem że dany proces musi wykonać sekwencję operacji by zmienić stan obiektu -- operacje nie są atomowe.

Na koniec stawiam do przemyślenia kwestie użycia wskaźników w kodzie wielowątkowym:
\begin{itemize}
\item Co stanie się po przekazaniu tego samego \code{MessagePtr} do 2 różnych instancji \code{MsgQueue}? czy będzie prawidłowo usunięty? Czy występuje ryzyko podwójnego usunięcia? Czy zliczanie referencji w \code{shared\_ptr} jest bezpieczne? czy jest to wydajne?
\item Co stanie się gdy kilka wątków zacznie modyfikować wiadomość udostępnianą przez \code{MsgQueue}?
\item Czy podobne kwestie są aktualne dla \code{weak\_ptr} i \code{unique\_ptr}
\end{itemize}

\paragraph{podsumowanie problemu}
W dostępie do zasobów należy uwzględnić następujące kwestie:
\begin{itemize}
\item Funkcje biblioteczne i systemowe niewspierające wielowątkowości.
\item Inicjację zasobu.
\item Odległość zasobu i blokady.
\item Wskaźniki i referencje.
\item Głębokość sekcji krytycznych.
\item Atomowość sekcji krytycznych.
\item Zachowanie spójność zasobu między sekcjami krytycznymi.
\item Unikanie wywołania zewnętrznego kodu wewnątrz sekcji krytycznych.
\end{itemize}

\section{Abstrakcja zasobu}\label{sec:resource}
Dlaczego zasoby sprawiają tyle problemów? Ponieważ są wyjątkowe i w przeciwieństiwe do informacji - niepomnażalne. Dla przykładu, ten artykuł można rozesłać do milionów urządzeń bez żadnych strat dla oryginalnej treści - treść będzie nieodróżnialna, jednak moje odręczne notatki są wyjątkowe, niepotarzalne, a przy próbie przesłania ich do każdego z czytelników prawdopodobnie nigdy nie dotarłyby do ostatniego w kolejce.

Dokładnie - w kolejce - przy próbie jednoczesnego dostępu moje notatki skończyłaby nie lepiej niż zdobycz w corocznej bitwie o karpia. Oczywiście my - programiści - z racji chronicznych niedostatków tkanki mięśniowej jesteśmy zmuszeni do zaniechania brutalnej rywalizacji na rzecz usystematyzowanych regół dostępu. Wg. etykiety czekamy aż zasób będzie dostępny - rezerwujemy go - używamy - i udostępniamy pozostałym gdy skończymy go używać. W skrócie stosujemy dostęp wykluczający - mutexy.

Łatwo powiedzieć trudniej zrobić. Pomimo że poleca je 9/10 programistów, mutexy posiadają zestaw wad:
\begin{itemize}
\item są duże -- std::mutex osiąga 40 bajtów
\item są na niskim poziomie abstrakcji -- musimy o nich cały czas pamiętać, zawsze i wszędzie.
\item powodują wzajemne wykluczenie wątków -- łatwo stają się wąskim gardłem aplikacji.
\item uwielbiają się zakleszczać na wszelkie możliwe sposoby.
\end{itemize}
Pomimo powyższych wad posiadają jednak 2 ogromne zalety - są proste w obsłudze i wszechstronne. Zrozumienie mechanizmu muteksu zajmuje kilka minut po których świeżo upieczony absolwent może zostać posłany na front walki o informatyzację kraju.

\paragraph{Resource}
\tname{Resource} jest krótkim szablonem który obmyśliłem jako metodę systematyzacji zarządzania zasobami przy pomocy muteksów. Szablon powstał na mój własny użytek podczas refaktoryzacji kodu jednej z bibliotek komunikacji sieciowej. Ma on na celu silnie związać zasób z osłaniającym go muteksem, zautomatyzować zajmowanie zasobu i zwalnianie zasobu, i stanowić ramę popychającą użytkownika w kierunku poprawnych metod zarządzania zasobami.

\includeCpp{../../examples}{Resource.hpp}

Kod jest krótki ale wariat. Jest relatywnie łatwy do zrozumienia i odpowiada wyłącznie za 1 rzecz - zarządzanie dostępem do zasobu.

Sama klasa szablonowa posiada 2 argumenty \cname{T} -- Przechowywany typ zasobu, i \cname{M} -- typ mutexu.

Dostępny jest tylko konstruktor forwardujący - aby zrozumieć tę konstrukcję należy zapoznać się z pojęciami \note{idealnego przekazywania}{\link{https://eli.thegreenplace.net/2014/perfect-forwarding-and-universal-references-in-c}} i \note{tzw. uniwersalnych referencji}{\link{https://www.youtube.com/watch?v=wQxj20X-tIU}} - w skrócie polega to na tym że nasza klasa \tname{Resource} jest wrapperem przezroczystym dla klasy przechowywanej.

Konstruktory kopiujące i porzypisania są jawnie usunięte z 3 powodów:
\begin{itemize}
\item Zasób to z definicji coś niereplikowalnego i zajmujacego unikalne miejsce w przestrzeni (rzeczywistej, czy pamięci komputera). Umożliwiając kopiowanie obiektów klasy \tname{Resource} złamałbym oczekiwania definiowane przez nazwę.
\item Z przyczyn praktycznych -- kopiowanie obiektu muteksu jest niemożliwe, a nawet gdyby było dla innych muteksów - sprawiałoby to liczne problemy związane ze spójnością stanu muteksu jak i przechowywanego obiektu.
\item Prostota -- jak zobaczymy w przykładach przechowywanie jednego odwołania co wymusza na użytkowniku zaniechania tworzenia pajęczej sieci powiązań.
\end{itemize}

Dostęp do obiektu jest realizowany przez akcesor \fname{\tname{critical\_section}}. Nazwa jest starannie dobrana by była odpowiednio długa, treściwa, groźna i odróżniała się od reszty metod. Dostęp odbywa się za pomocą podanego w argumencie wizytatoru który jest wywoływany w postaci sekcji krytycznej - muteks jest automatycznie zajmowany przez \tname{lock\_guard}, tóż przed wejściem wywołaniem argumentu i zwalniany tóż po powrocie. Użycie lock guard gwarantuje zwolnienie także w przypadku wyrzucenia wyjątku. Dodatkową zaletą użycia \tname{lock\_guard} jest możliwość zdefiniowania własnej implementacji dla customowych muteksów, a tym samym brak wymagań stawianych na klasę muteksu.

Powodem napisania tego prostego szablonu była potrzeba stworzenia interface'u biblioteki wywoływanego w trybie asynchronicznym, gdy każde z wywołań mogło odwoływać się do tego samego zestawu zasobów z czego każde wywołanie mogło operować na więcej niż jednym zasobie. Ryzyko zakleszczeń i spowolnienia związane ze zbyt długimi sekcjami krytycznymi sprawiły że chciałem usystematyzować rozdział sekcji krytycznych.

\onecolumn
\theendnotes
\end{document}
