\documentclass[titlepage,twocolumn,a4paper,10pt,fleqn,leqno]{article}

\usepackage[utf8]{inputenc}
\usepackage[T1]{fontenc}
\usepackage{polski}
\usepackage{titlesec}
\usepackage{wrapfig}
\usepackage{listings}
\usepackage{blindtext}
\usepackage{xcolor}

\definecolor{red}{rgb}{1,0,0}
\definecolor{codegreen}{rgb}{0,0.6,0}
\definecolor{codegray}{rgb}{0.5,0.5,0.5}
\definecolor{codepurple}{rgb}{0.58,0,0.82}
\definecolor{backcolour}{rgb}{0.95,0.95,0.95}

\lstdefinestyle{cpp-style}{
    backgroundcolor=\color{backcolour},
    commentstyle=\color{codegreen},
    keywordstyle=\color{magenta},
    numberstyle=\tiny\color{codegray},
    stringstyle=\color{codepurple},
    basicstyle=\ttfamily\scriptsize,
    breakatwhitespace=false,
    breaklines=true,
    captionpos=b,
    keepspaces=false,
    numbers=left,
    numbersep=1pt,
    showspaces=false,
    showstringspaces=false,
    showtabs=false,
    tabsize=2
}

\newcommand\includeCpp[2]{\noindent\lstinputlisting[language=C++,style=cpp-style,captionpos=t,title=\scriptsize#2]{#1/#2}}
\newcommand\includeText[2]{\noindent\lstinputlisting[captionpos=t,style=cpp-style,title=\scriptsize#2]{#1/#2}}

\newcommand\inlineBash[1]{\colorbox{backcolour}{\lstinline[language=bash,style=cpp-style,frame=lines]{$> #1}}}

\newcommand\inlineCpp[1]{\colorbox{backcolour}{\lstinline[language=C++,style=cpp-style]{#1}}}

\titleformat{\section}
  {\normalfont\large\bfseries}{\thesection.}{1em}{}
\titleformat{\subsection}
  {\normalfont\normalsize\itshape}{\thesubsection.}{1em}{}
\titleformat{\subsubsection}
            {\normalfont\normalsize\itshape}{\thesubsubsection.}{1em}{}

\usepackage{relsize,endnotes,xurl,hyperref}

\renewcommand\UrlFont{\rmfamily}

\newcommand{\note}[2]{#1\endnote{#2}}

\newcommand{\link}[1]{\url{#1}}

\newcommand{\mail}[1]{\href{mailto:#1}{#1}}

\newcommand{\Rsign}[1]{\protect\hspace{-.15em}\protect\raisebox{.4ex}{\smaller\smaller\smaller\textbf{#1}}}
\newcommand{\Cpp}[1]{\mbox{C\Rsign{+}\Rsign{+}\protect\hspace{-.15em}#1}}

\makeatletter
\renewcommand\paragraph{%
  \@startsection{paragraph}{4}{0mm}%
                {-\baselineskip}%
                {0.5\baselineskip plus 0.2\baselineskip minus 0.2\baselineskip}%
                {\normalfont\normalsize\bfseries}}
\makeatother

\newcommand\editNote[2]{#1\textit{\color{red}{<<#2>>}}}

\newcommand\code[1]{\texttt{#1}}
\newcommand\cname[1]{\code{#1}}
\newcommand\fname[1]{\code{#1()}}
\newcommand\tname[1]{\code{#1<>}}

\usepackage{multirow,graphicx}

\newcommand\setTitle[1]{\title{#1}\def\titleText{#1}}
\newcommand\setTitlePhoto[1]{\def\titlePhoto{#1}}
\newcommand\setTitleDescription[1]{\def\titleDescription{#1}}
\newcommand\setTitleCatchphrases[1]{\def\titleCatchphrases{#1}}
\newcommand\setAuthor[1]{\author{#1}\def\authorName{#1}}
\newcommand\authorPhoto[1]{\def\authorPhoto{#1}}
\newcommand\authorPosition[1]{\def\authorPosition{#1}}
\newcommand\authorWebpage[1]{\def\authorWebpage{#1}}
\newcommand\authorEmail[1]{\def\authorEmail{#1}}

\newcommand\getAuthor{\authorName}
\newcommand\getTitle{\titleText}
\newcommand\getTitleCatchphrases{\titleCatchphrases}

\newcommand\printAuthor{%
\setlength\tabcolsep{0pt}%
\begin{tabular}{p{.89\textwidth}p{.10\textwidth}p{.01\textwidth}}
\Large\getAuthor & \multirow{4}{*}{\includegraphics[height=4.1\baselineskip]{\authorPhoto}} &\\
\footnotesize\authorPosition & &\\
\footnotesize\authorWebpage & & \\
\footnotesize\mail{\authorEmail} & &\\
\end{tabular}
}
\newcommand\printTitle{\center{\mbox{\Huge\bf{\getTitle}}}}
\newcommand\printCatchphrases{\footnotesize{\getTitleCatchphrases}}
\newcommand\printTitlePhoto{\includegraphics[width=.8\textwidth]{\titlePhoto}}
\newcommand\printTitleDescription{\framebox[\linewidth]{%
\parbox{0.9\linewidth}{\vspace{10pt}\textit{\titleDescription}\vspace{10pt}}%
}}

\renewcommand\maketitle{\noindent
\printAuthor
\rule{\textwidth}{1pt}
\vspace{2cm}
\printTitle\\
\vspace{1cm}
\center{\printTitlePhoto}\\
\vspace{1cm}
\center{\printTitleDescription}\\
\center{\printCatchphrases}
}


\setTitle{Współdzielenie zasobów}
\setTitleCatchphrases{\Cpp, Przetwarzanie współbieżne, Struktura programu}
\setTitlePhoto{images/pizza}
\setTitleDescription{Artykuł prezentuje wybrane metody dostępu do zasobów w przykładach wzorowanych na kodzie produkcyjnym. Autor omawia problematykę, a następnie wskazuje metody usystematyzowania dostępu do elementów współdzielonych.}

\setAuthor {Bartosz Charuza}
\authorPhoto {images/author}
\authorPosition {Regular \Cpp{} Software Developer}
\authorProject {Alstom: Ebiscreen2000}
\authorWebpage {\link{https://github.com/bcharuza}}
\authorEmail {bcharuza@yahoo.com}

\hypersetup{
  pdfauthor=\getAuthor,
  pdftitle=\getTitle,
  pdfpagemode=None,
  urlbordercolor=black,
  pdfborderstyle={/S/U/W 0.3}}


\begin{document}
\begin{titlepage}
\maketitle
\end{titlepage}
\section{Wstęp}\label{sec:introduction}
\paragraph{Dlaczego napisałem ten artykuł?}
Historia tego artykułu miała początek pewnego pięknego październikowego ranka, krótko po załączeniu komputera otrzymałem radosne przywitanie. ,,Hello. How are you?''.\\
Poniżej czaiły się 3 kropki grozy\\
...\\
\\
,,Sometimes X doesn't work'' - oto mój bilet na autystyczne randez-vous z tysiącami linii kodu.
,,Sometimes'' jest szczególnie złowieszczym słowem, gdyż ujawnia wyścig - teraz będę  godzinami gonił czerwoną kropkę drwiącego losu. Po 2 dniach galerniczej przygody odkryłem skarb! Wyścig był powodowany zajęciem niewłaściwego muteksu. Nastąpił przełom - \note{niczym oporna drukarka, która zmieniła losy świata}{\link{https://en.wikipedia.org/wiki/Richard\_Stallman\#Events\_leading\_to\_GNU}}, ten niedziałający muteks zmobilizował mnie do przepisania biblioteki i napisania serii artykułów.

\paragraph{Kto jest adresatem?}
Programiści \Cpp{}, od juniora który może znaleźć inspirację, po seniora który może chcieć wymagać podobnych rozwiązań w swoim projekcie. Zakładam że czytelnik posiada podstawową znajomość \Cpp{14} i wiedzę z zakresu przetwarzania wielowątkowego z użyciem mutexów biblioteki standardowej.

\paragraph{Jaki jest zakres artykułu?}
Publikacja prezentuje wzorzec \tname{Resource} i przypadki użycia. Omawiane będą kontrprzykłady, zagadnienia inicjacji i dostępu do zasobów. Nie opisuję narzędzi, ani podstaw wielowątkowości. Większość poruszanych tematów będzie związana z wykorzystaniem wspomnianego wzorcu.

Prezentowane metody sprawdzone są w aplikacjach klasy desktop przy użyciu mechanizmów biblioteki standardowej do kilkudziesięciu wątków. Nie będą poruszane rozwiązania dla aplikacji serwerowych, czy o wysokich wymaganiach wydajnościowych w których nie mam doświadczenia. Artykuł nie ma też zastosowania dla problematyki massive-parallelizm.

\paragraph{Struktura artykułu}
Próbowałem napisać ten artykuł w standardowej akademickiej strukturze problem-analiza-rozwiązanie. Skutkowało to zbyt rozciągłą formą wraz dojściem nowych wątków, dywagacji i przypadków szczególnych.

Zdecydowałem się zastosować formułę rozwiązanie-zastosowanie, gdzie przypadki użycia są podzielone na krótkie pary kontrprzykład-przykład. Mam nadzieję że pozwoli to zredukować rozmiar do minimum i dotrwać czytelnikowi do końca artykułu.

\section{Przygotowanie merytoryczne}\label{sec:reader-profile}
\paragraph{Oznaczenia}
\begin{itemize}
\item \cname{Nazwy\_systemowe}
\item \fname{Nazwy\_funkcji}
\item \tname{Nazwy\_szablonów}
\end{itemize}
\paragraph{Zasób}
\note{Zasób to coś o ograniczonej dostępności.}{\link{https://en.wikipedia.org/wiki/System\_resource}} Zasoby w przeciwieństwie do informacji/wartości posiadają tożsamość. Zasobem może być urządzenie zewnętrzne, pamięć, czas procesora, a nawet obiekt.

\paragraph{Typ}
\note{Typ określa zestaw dozwolonych wartości i operacji obiektu.}{Bjarne Stroustrup: ,,Język \Cpp. Kompendium wiedzy.''}

\paragraph{Wartość}
\note{Wartość to zbiór bitów interpretowanych zgodnie z typem}{Bjarne Stroustrup: ,,Język \Cpp. Kompendium wiedzy.''}

\paragraph{Obiekt}
\note{Obiekt to miejsce w pamięci, w którym przechowywana jest wartość jakiegoś typu}{Bjarne Stroustrup: ,,Język \Cpp. Kompendium wiedzy.''}

\paragraph{Sekcja krytyczna}
\note{Sekcją krytyczną jest zestaw instrukcji chroniony przed jednoczesnym wywołaniem względem danego zasobu}{\link{https://en.wikipedia.org/wiki/Critical\_section}}

\paragraph{Znajomość \Cpp{14}}
Większość kodu będzie napisana w \Cpp{14}. Czytelnik znający \Cpp{11} nie powinien widzieć różnicy. Czytelnik znający \Cpp{03} powinien być w stanie zrozumieć treść. Starsze standardy różnią się zbyt mocno aby artykuł mógł być łatwo przyswajalny.

\paragraph{Przetwarzanie współbieżne}
Znajomość mutexów i podstaw wielowątkowości biblioteki standardowej \Cpp{} powinna być wystarczająca.

\paragraph{Kompilacja i linkowanie}
By zrozumieć niektóre problemy czytelnik powinien znać proces kompilacji i linkowania plików źródłowych do binarnych.

\paragraph{Programowanie szablonowe}
Część kodu jest kodem szablonowym. Do pełnego zorumienia implementacji należy rozumieć podstawy programowania szablonowego i narzędzia szablonowe biblioteki standardowej.

\section{Środowisko testowe}\label{sec:environment}
\begin{tabular}{l r}
  Kompilator:   & GCC v9.3\\
  System:       & Ubuntu 20.04.2 LTS\\
  Architektura: & x86\_64
\end{tabular}

\section{Abstrakcja zasobu}\label{sec:resource}
Dlaczego zasoby sprawiają tyle problemów? Ponieważ są wyjątkowe i w przeciwieństiwe do informacji - niepomnażalne. Dla przykładu, ten artykuł można rozesłać do milionów urządzeń bez żadnych strat dla oryginalnej treści - treść będzie nieodróżnialna, jednak moje odręczne notatki są wyjątkowe, niepotarzalne, a przy próbie przesłania ich do każdego z czytelników prawdopodobnie nigdy nie dotarłyby do ostatniego w kolejce.

Dokładnie - w kolejce - przy próbie jednoczesnego dostępu moje notatki skończyłaby nie lepiej niż zdobycz w corocznej bitwie o karpia. Oczywiście my - programiści - z racji chronicznych niedostatków tkanki mięśniowej jesteśmy zmuszeni do zaniechania brutalnej rywalizacji na rzecz usystematyzowanych regół dostępu. Wg. etykiety czekamy aż zasób będzie dostępny - rezerwujemy go - używamy - i udostępniamy pozostałym gdy skończymy go używać. W skrócie stosujemy dostęp wykluczający - mutexy.

Łatwo powiedzieć trudniej zrobić. Pomimo że poleca je 9/10 programistów, mutexy posiadają zestaw wad:
\begin{itemize}
\item są duże -- std::mutex osiąga 40 bajtów
\item są na niskim poziomie abstrakcji -- musimy o nich cały czas pamiętać, zawsze i wszędzie.
\item powodują wzajemne wykluczenie wątków -- łatwo stają się wąskim gardłem aplikacji.
\item uwielbiają się zakleszczać na wszelkie możliwe sposoby.
\end{itemize}
Pomimo powyższych wad posiadają jednak 2 ogromne zalety - są proste w obsłudze i wszechstronne. Zrozumienie mechanizmu muteksu zajmuje kilka minut po których świeżo upieczony absolwent może zostać posłany na front walki o informatyzację kraju.

\paragraph{Resource}
\tname{Resource} jest krótkim szablonem który obmyśliłem jako metodę systematyzacji zarządzania zasobami przy pomocy muteksów. Szablon powstał na mój własny użytek podczas refaktoryzacji kodu jednej z bibliotek komunikacji sieciowej. Ma on na celu silnie związać zasób z osłaniającym go muteksem, zautomatyzować zajmowanie zasobu i zwalnianie zasobu, i stanowić ramę popychającą użytkownika w kierunku poprawnych metod zarządzania zasobami.

\includeCpp{../../examples}{Resource.hpp}

Kod jest krótki ale wariat. Jest relatywnie łatwy do zrozumienia i odpowiada wyłącznie za 1 rzecz - zarządzanie dostępem do zasobu.

Sama klasa szablonowa posiada 2 argumenty \cname{T} -- Przechowywany typ zasobu, i \cname{M} -- typ mutexu.

Dostępny jest tylko konstruktor forwardujący - aby zrozumieć tę konstrukcję należy zapoznać się z pojęciami \note{idealnego przekazywania}{\link{https://eli.thegreenplace.net/2014/perfect-forwarding-and-universal-references-in-c}} i \note{tzw. uniwersalnych referencji}{\link{https://www.youtube.com/watch?v=wQxj20X-tIU}} - w skrócie polega to na tym że nasza klasa \tname{Resource} jest wrapperem przezroczystym dla klasy przechowywanej.

Konstruktory kopiujące i porzypisania są jawnie usunięte z 3 powodów:
\begin{itemize}
\item Zasób to z definicji coś niereplikowalnego i zajmujacego unikalne miejsce w przestrzeni (rzeczywistej, czy pamięci komputera). Umożliwiając kopiowanie obiektów klasy \tname{Resource} złamałbym oczekiwania definiowane przez nazwę.
\item Z przyczyn praktycznych -- kopiowanie obiektu muteksu jest niemożliwe, a nawet gdyby było dla innych muteksów - sprawiałoby to liczne problemy związane ze spójnością stanu muteksu jak i przechowywanego obiektu.
\item Prostota -- jak zobaczymy w przykładach przechowywanie jednego odwołania co wymusza na użytkowniku zaniechania tworzenia pajęczej sieci powiązań.
\end{itemize}

Dostęp do obiektu jest realizowany przez akcesor \fname{\tname{critical\_section}}. Nazwa jest starannie dobrana by była odpowiednio długa, treściwa, groźna i odróżniała się od reszty metod. Dostęp odbywa się za pomocą podanego w argumencie wizytatoru który jest wywoływany w postaci sekcji krytycznej - muteks jest automatycznie zajmowany przez \tname{lock\_guard}, tóż przed wejściem wywołaniem argumentu i zwalniany tóż po powrocie. Użycie lock guard gwarantuje zwolnienie także w przypadku wyrzucenia wyjątku. Dodatkową zaletą użycia \tname{lock\_guard} jest możliwość zdefiniowania własnej implementacji dla customowych muteksów, a tym samym brak wymagań stawianych na klasę muteksu.

Powodem napisania tego prostego szablonu była potrzeba stworzenia interface'u biblioteki wywoływanego w trybie asynchronicznym, gdy każde z wywołań mogło odwoływać się do tego samego zestawu zasobów z czego każde wywołanie mogło operować na więcej niż jednym zasobie. Ryzyko zakleszczeń i spowolnienia związane ze zbyt długimi sekcjami krytycznymi sprawiły że chciałem usystematyzować rozdział sekcji krytycznych.

\section{Przykłady}\label{sec:examples}
Klasa \tname{Resource} pomogła mi w zwalczaniu błędów występujących w poprawianym kodzie:
\begin{itemize}
\item brak pokrycia zakresów mutexu i zasobu.
\item długie sekcje krytyczne.
\item jednoczesne zajmowanie wielu zasobów.
\item wyścigi danych
\item \note{wyścigi interface'u}{Anthony Williams ,,Język C++ i przetwarzanie współbieżne w akcji''}
\end{itemize}

\paragraph{CfgReader}
Pierwszym przykładem, który chcę omówić jest klasa \cname{CfgReader} -- silnie zredukowany przedstawiciel kodu który spotkałem w produkcji. Działanie jest banalne - po wczytaniu konfiguracji za pomocą metody \fname{LoadCfg} aktualizowane są pola statyczne klasy.
\includeCpp{../../examples}{CfgReader.hpp}
\includeCpp{../../examples}{CfgReader.cpp}
Można zauważyć następujące błędy:
\begin{itemize}
\item Pola mogą być odczytywane i zapisywane asynchronicznie doprowadzając do wyścigu danych.
\item \fname{LoadCfg} ma zakres instancji a modyfikowane pola są statyczne - to mylące i niebezpieczne.
\item Nie da się określić czy pola zostały zainicjowane czy jeszcze nie.
\item Przeładowanie konfiguracji nie jest operacją atomową.
\end{itemize}
Powyższy kod wymusza by konfiguracja była inicjowana jak najwcześniej podczas rozruchu aplikacji i by była niezmienna w dalszym ciągu działania programu - nic jednak nie chroni użytkownika przed wielokrotnym wywołaniem \fname{LoadCfg}. Korzystanie z pól statycznych klasy przed wywołaniem \fname{LoadCfg} musi być pilnowane przez programistę. Po inicjacji konfiguracja nie może być już przeładowana, i jest sprowadzona do zbioru wartości. Nie jesteśmy nawet w stanie kontrolować przypadkowej modyfikacji.

Jak użycie \tname{Resource} jest w stanie poprawić kod naszej klasy? 

\includeCpp{../../examples}{CfgReader-fix.hpp}
\includeCpp{../../examples}{CfgReader-fix.cpp}
Co zmieniłem?
Przeniosłem zarządzanie zasobem na wrapper \tname{Resource} i udostępniam przez funkcję globalną. Dostęp jest kontrolowany wewnętrznie. Pola statyczne zmieniłem na mapę - do przechowywania stanów współdzielonych jak konfiguracja mapy sprawdzają się najlepiej.

Funkcja \fname{LoadCfg} otrzymuje dostęp do mapy przez funkcję dostępową \fname{getCfg}, która udostępnia wspomnianą mapę owiniętą wewnątrz \tname{Resource}. Schemat aktualizacji konfiguracji jest standardowy - tworzymy lokalną strukturę, a następnie przypisujemy ją do zasobu w pojedynczym szybkim przypisaniu ograniczając czas wykonania sekcji krytycznej do minimum.

Dlaczego użyłem funkcji dostępowej zamiast umieścić zasób w zmiennej globalnej?
Ponieważ standard definiuje kolejność inicjacji zmiennej globalnej w zależności od pierwszeństwa jej wystąpienia, co oznacza że jest ona zależna od kolejności kompilacji/linkowania. Aby to objeść użyłem prostego rozwiązania - funkcji globalnej. Standard wymaga by pola statyczne były inicjowane przed jej wywołaniem tylko 1 raz - nawet w kodzie wielowątkowym. Kompilator automatycznie ustali kolejność inicjacji. Dodatkowo pozwala to na prostszą refaktoryzację i testowanie kodu.

\paragraph{Logger}
\cname{Logger} reprezentuje typowy kod C przełożony na grunt \Cpp{}. Jest to silnie zredukowany przykład wzorowany na popularnych leniwych logerach-samoróbkach, często przeplatanych niezależnymi operacjami wyjściowymi jak \fname{printf}, czy \code{std::cout} w innych miejscach programu. Jest to prawdziwy kłopot dla kodu wielowątkowego.
\includeCpp{../../examples}{Logger.hpp}
\includeCpp{../../examples}{Logger.cpp}
W powyższym kodzie są wykorzystywane funkcje systemowe które nie dają gwarancji bezpieczeństwa lub gwarancja jest zależna od implementacji biblioteki jak \fname{vfprintf}. \note{W systemach zgodnych z \mbox{POSIX} funkcje rodziny \fname{printf} są bezpieczne, jednak \fname{ctime} już nie musi}{\link{https://pubs.opengroup.org/onlinepubs/9699919799/functions/V2\_chap02.html}}. Dla Windows nie spotkałem się z taką gwarancją. Używając biblioteki C trzeba szczególnie zwracać uwagę czy wybrany element jest przystosowany do użycia w środowisku wielowątkowym. Dostęp do konfiguracji jest niechroniony i może powodować liczne błędy wyścigu do danych.

Jak ta kwestia wygląda w \Cpp{} ?
Dla porównania wg. \note{propozycji p0053r7}{\link{https://isocpp.org/files/papers/p0053r7.pdf}} standard zapewnia bezpieczeństwo dla strumieni wyjściowych takich jak \vname{std::cout}. Bezpieczeństwo jest gwarantowane tylko dla pojedynczej operacji, więc \code{cout<<A<<B} może zostać przerwane przez inny proces między wywołaniem \code{<\,<A} i \code{<\,<B}. Należy też uważać na starsze wersje biblioteki \lib{stdc++}, która mogła niepoprawnie obsługiwać wielowątkowe operacje na obiektach strumieni. Standard \Cpp{20} udostępnia już wrapper \cname{basic\_osyncstream}.

Spróbujmy przeprojektować powyższy kod, tak by zachować wygodę \fname{vfprintf} i zagwarantować bezpieczeństwo wielowątkowe.

\includeCpp{../../examples}{Logger-fix.hpp}
\includeCpp{../../examples}{Logger-fix.cpp}
Co zmieniono?
Funkcję \fname{\tname{trace}} kompletnie przepisałem, tak by używała tylko biblioteki \Cpp{}. Wyświetlenie wiadomości odbywa się w pojedynczym wywołaniu \fname{ostream::operator <\,<} by mieć pewność że wydruk jest bezpieczny ze względu na wątki. Zapełnianie buforu \cname{oss} odbywa się w pamięci lokalnej, więc wszelkie interakcje z zasobem \vname{std::cout} są ograniczone do minimum.

Funkcja \fname{trace\_sysinfo} jest odpowiedzialna za zapisanie nagłówku wiadomości i przeniesiono ją do \file{Logger-fix.cpp} by ograniczyć zależności \file{Logger-fix.hpp}.

\fname{\tname{trace\_impl}} odpowiada za rekurencyjne wywołania szablonowe o zmiennej liczbie argumentów. Jest to rozwiązanie dające możliwości variadic functions znanych z C i jednocześnie zachowujące ścisłą kontrolę typów znaną z \Cpp{}. Dodatkowo użytkownik może definiować własne przeciążenia \fname{\tname{trace\_print}} by dodać własną obsługę typów złożonych. Funkcje pomocnicze umieszczono w tzw. anonimowej przestrzeni nazw, by uniemożliwić dostęp do nich spoza pliku.

Warto zauważyć że dostęp do konfiguracji odbywa się przez \fname{getCfg} atomowo w krótkiej sekcji krytycznej. Obsługa sekcji krytycznej jest niemal niezauważalna dla użytkownika kodu i objawia się użyciem lambdy.

\paragraph{MessageHandler}
Ostatnim przykładem jest klasa \cname{MessageHandler}, która też jest inspirowana kodem produkcyjnym i przedstawia częste bolączki niewłaściwie zaprojektowanego układu klas.
\includeCpp{../../examples}{MessageHandler.hpp}
\includeCpp{../../examples}{MessageHandler.cpp}
W powyższym kodzie klasa \cname{MsgHandler} pełni funkcję centralnej koordynacji przepływu wiadomości. Zawiera jeden globalny interface we/wy \cname{MQClient} z którym związuje się przez wymianę adresów. Już na tym etapie widać pierwszą pułapkę - adres nic nie mówi o czasie życia obiektu i gdy jeden z obiektów przestanie istnieć narażamy się na naruszenie ochrony pamięci. Drugim trudniej zauważalnym problemem jest pełzająca spagettyfikacja kodu - tworzenie takich odniesień 1-1 powoduje że adres zasobu będzie przekazywany między instancjami klienckimi. Modyfikacja i zrozumienie zależności takiego kodu będą z czasem coraz trudniejsze. Aby tego uniknąć zasoby powinny być umieszczane w globalnych repozytoriach.

Co gorsze połączenia wcale nie muszą być symetryczne. Po wielokrotnym wywołaniu \fname{setClient} dla różnych instancji \cname{MsgHandler} otrzymamy wiele instancji \cname{MQClient} wskazujących na różne \cname{MsgHandler}, które z kolei wskazują na ostatnią powiązaną instancję \cname{MQClient}. To nie może skończyć się dobrze.

Poza we/wy klasa \cname{MsgHandler} zarządza także mechanizmem subskrypcji które polegają na wywołaniu callbacku po otrzymaniu nowej wiadomości. Subskrypcje są współdzielone między instancjami \cname{MsgHandler}, a subskrybenci są informowani kolejno wewnątrz metody \fname{NotifyAll} - to powoduje że zablokowanie jednego opóźnienia generowane przez każdego z odbiorców się kumulują. Pomijając możliwe zakleszczenia powodowane wywołaniem kodu klienckiego pewne są też długie czasy wykonania w wątku wywołującym. Przeniesienie obsługi notyfikacji do osobnego wątku mogłoby poprawić responsywność, jednak jest to obecnie spychane na użytkownika klasy.

Ostatnim z obowiązków klasy \cname{MsgHandler} jest zarządzanie kolejką wiadomości która także dzielona między instancjami. Sama klasa \cname{MsgQueue} jest zrealizowana prawidłowo, jednak nie jest ona przystosowana do kodu wielowątkowego. Co prawda \cname{MsgQueue} jest chroniona przez \cname{MsgHandler}, jednak ten zezwala na tzw. wyścig interface'u. Wyścig interface'u polega na tym że klasa zakłada niejawną transakcję - tj. że klient musi pamiętać jaki jest stan zasobu między 2 kolejnymi operacjami. Zapamiętany stan może być w międzyczasie zmodyfikowany przez inny wątek. Przykładowo mając 2 wątki \code{A} i \code{B} i współdzielony obiekt \code{MsgHandler x}:

\begin{tabular}{l | l}
  \code{A} & \code{B}\\
  \hline
  \code{if(x.isEmpty())} & \code{if(x.isEmpty())}\\
  \code{  x.sendTop()} & \code{  x.sendTop()}
\end{tabular}

\noindent
Dla kolejności AABB i BBAA powyższy kod może być wykonany poprawnie, jednak dla ABAB, BABA, ABBA i BAAB może doprowadzić do załamania aplikacji (gdy w kolejce jest 1 element), pomimo że zarówno \fname{isEmpty} jak i \fname{sendTop} są wykonywane jako sekcje krytyczne.

Kończąc, pozostaje zauważyć że jak na ironię jedynym elementem który nie jest dzielony między instancjami jest mutex, który powinien chronić elementy współdzielone przed jednoczesnym dostępem. Obecnie nie spełnia on swojej roli.

Powyższy kod skompiluje się, jednak klasa nie będzie działać. Błąd jest prosty, więc proponuję by czytelnik spróbował go odszukać, aby zrozumieć dlaczego zdecydowałem się napisać \tname{Resource}.

Strukturę tego programu można szybko poprawić za pomocą globalnych repozytoriów zasobów i wrapperu \tname{Resource}. Pytaniem otwartym jest - co potraktujemy jako zasób? Aby to zrobić trzeba przeanalizować przypadki użycia i przepływ informacji. Wiemy że:
\begin{itemize}
\item jedyną klasą posiadającą dane współdzielone jest \cname{MsgHandler}.
\item Każda z instancji \cname{MQClient} może komunikować się z jedną instancją \cname{MsgHandler}
\item Każda z instancji \cname{MsgHandler} musi komunikować się z ostatnią skonfigurowaną \cname{MQClient}.
\end{itemize}

Wyłania się z tego obraz systemu muksującego (nie mylić z mutexem) z buforowaniem. System muksujący kieruje wiele sygnałów wejściowych na jedno wyjście. Wobec tego powinniśmy posiadać możliwość tworzenia nieograniczonej ilości instancji \cname{MQClient} i posiadać możliwość zarejestrowania 1 instancji jako wyjścia \cname{MsgHandler}. Moim zdaniem taki hub nie powinien móc współdzielić pamięci między swoimi instancjami więc subskrypcje i kolejka powinny być przechowywane lokalnie. Metody \cname{MsgHandler} będą wykonywane współbieżnie, więc instancje tej klasy traktować należy jako zasób. Z kolei \cname{MQClient} chcemy używać lokalnie dla wątku. Co jednak się stanie gdy zarejestrujemy obiekt w dwóch \cname{MsgHandler}? Jak poinformować \cname{MsgHandler} że obiekt przestał istnieć? Dostęp do \cname{MQClient} także należałoby przeprojektować.
\includeCpp{../../examples}{MessageHandler-fix.hpp}
\includeCpp{../../examples}{MessageHandler-fix.cpp}
Co się zmieniło? Najważniejszą zmianą jest dodanie \fname{initializeMsgHandler} i \fname{accessMsgHandler} które przyjmują formę prostego repozytorium dla obiektów \cname{MsgHandler}.

W \cname{MQClient} zmieniłem sposób przechowywania muxu z prostego wskaźniku na \tname[Resource]{shared\_ptr}. Wrapperem \tname{Resource} chronimy osobno wejście i wyjście narażone na współbieżny dostęp z \cname{MsgHandler} - zauważ jak naturalne staje się odseparowanie sekcji krytycznych w \fname{MQClient::receive}. Pobranie instancji \cname{MsgHandler} odbywa się tylko z repozytorium zasobów. Warto nadmienić, że wykorzystano tu bezpieczeństwo zliczania odniesień \tname{shared\_ptr} dla wywołań współbieżnych. Dostępem administruje zaś wrapper \tname{Resource}. Alternatywnym rozwiązaniem może być też przechowywanie nazwy/id instancji muxu i każdorazowe pobieranie instancji z repozytorium.

\cname{MsgHandler} wchłonął klasę \cname{Subscription} która wyłącznie obciążała użytkownika - teraz operacje nadawania unikalnego ID odbywają się we wnętrzu klasy w \fname{getNewId}. Wszystkie pola klasy są polami instancji dzięki czemu unikamy konieczności pilnowania dostępu do nich.

Z \cname{MsgHandler} usunięto mutex instancji. Nie jest to rozwiązanie czysto pozytywne - ma wady i zalety. Wadą jest porzucenie kontroli nad dostępem do klasy - przez co dla środowiska wielowątkowego musi być zawsze otoczona przez wrapper \tname{Resource}. Niewątpliwą zaletą jest jednak brak konieczności zajmowania się dostępem wielowątkowym. Usunięto mutex, którego nie trzeba pilnować w kodzie i prawdopodobieństwo wystąpienia wyścigu interface'u zostało znacznie zredukowane.

Oczywiście znacznie lepszym rozwiązaniem byłoby przebudowanie prezentowanego wcześniej \cname{MsgHandler} tak aby klasa realizowała interface atomowy kontrolowany przez własne mutexy czy operacje bezkolizyjne. Klasy takie muszą zostać jednak przygotowane przez doświadczonego programistę specjalizującego się w kodzie współbieżnym, który zastosowałby zresztą inne rozwiązania architektoniczne.

%\section{Opis problemu}\label{sec:problem-desc}
Za punkt wyjścia przyjmiemy spreparowany omówiony poniżej, które można znaleść w katalogu \code{./examples/problem} \note{projektu}{\link{https://github.com/bcharuza/resource-management/}}. Przykłady są syntezą kodu produkcyjnego zastanego w kilku projektach z którymi pracowałem.

\paragraph{main}
Main wskazuje przykładowy kod programu. Dobrym ćwiczeniem dla czytelnika byłoby wskazanie kilku poważnych błędów zwązanych z nieprawidłowym użyciem przedstawionych uprzednio struktur.
\includeCpp{../../examples/problem}{main.cpp}
Podstawowym problemem tego kodu jest to że uprzednie struktury zezwoliły ich użytkownikowi na ich nieprawidłowe użycie. Interface jest nieprawidłowo skonstruowany. Co gorsze po wykonaniu pierwszego testu za pomocą \inlineBash{ctest -R problem-test1} (lub \inlineBash{<problem-binary> test1.cfg} gdy kompilujemy bez CMake) aplikacja się zawiesza.
\includeText{../../tests/}{test1.cfg}
\includeText{../../tests/}{test1-in.txt}
Zawieszenie się aplikacji jest spowodowane podwójnym zajęciem muteksu klasy. Każda z metod zajmuje muteks instancji by zablokować klasę. W przypadku gdy \code{std::mutex} jest zajmowany poraz 2 w tym samym wątku następuje zakleszczenie. Jest to znacznie trudniejsze do wykrycia gdy muteks jest statyczny, lub co gorsze - współdzielony w pamięci wolnej. Można to rozwiązać zamieniając \code{std::mutex} na \code{std::recursive\_mutex} \note{jednak jego użycie uznaje się za symptom źle zaprojektowanego współdzielenia zasobów}{Anthony Williams ,,Język \Cpp{} i przetwarzanie współbieżne w akcji''} - w skrócie jeśli pozwalamy na zakleszczenie na pojedynczym muteksie, prawdopodobnie kod jest też wrażliwy na zakleszczenie w scenariuszach z wieloma muteksami, a sekcje krytyczne są zbyt rozbudowane.

Podwójne blokowanie można naprawić przez zamianę definicji \inlineCpp{using mtx\_t = std::mutex} na \inlineCpp{using mtx\_t = std::recursive\_mutex} w pliku \code{MessageHandler.hpp}. Kod nadal będzie posiadał wyścig wewnątrz \code{ping\_race()}, ale logi początkowe będą możliwe do odczytania. Wyścig w \code{pingrace()} jest spowodowany umieszczeniem muteksu i zasobów w różnych zakresach - muteks jest lokalny dla instancji, zaś zasoby są współdzielone przez obiekty.
Po ponownym uruchomieniu testu otrzymamy następujący wynik:
\begin{itemize}
\item \code{test1-out.txt}: pusty lub wypełniony tylko początkowymi wiadomościami. Nie widać tego w kodzie - jednak jak opisano wcześniej, globalna instancja \code{s\_handler} i lokalna wątku \code{handler1} wpływają na siebie.
\item \code{log}: zawiera wiadomość diagnostyczną i zdublowane wiadomości które miały trafić do \code{test1-out.txt}. Dublowanie wiadomości jest także spowodowane tym że \code{sender} został 2-krotnie zasubskrybowany we współdzielonej przestrzeni pomimo że wygląda jakby był zasubskrybowany dla obiektu.
\item \code{konsola}: Kolejne błędy w kodzie można dostrzec po sprawdzeniu wyjścia konsoli.
  \begin{itemize}
  \item \code{((null)[140624307808064])[INFO] Initialized MsgHandler}: trace został wywołany zanim zainicjowano zmienną statyczną \code{CfgReader::s\_sysName}.
    \item \code{(Sys1[140250092123968])[INFO] msgInput:  śčĄţ} Błędy są spowodowane nieprzekazaniem właściwego typu do trace. Kompilator nie był w stanie tego wychwycić.
  \end{itemize}
\end{itemize}

\code{main.cpp} posiada jeszcze jeden błąd trudniejszy do uchwycenia - wyścig instrukcji podczas usuwania elementów kolejki - pomiędzy wywołaniem \code{top()} i \code{remove()} inny proces może też wykonać \code{set()} a wtedy \code{remove()} spowoduje usunięcie wiadomości niebędącej już na szczycie. Jest to spowodowane otwarciem nieformalnej traksakcji -- założeniem że dany proces musi wykonać sekwencję operacji by zmienić stan obiektu -- operacje nie są atomowe.

Na koniec stawiam do przemyślenia kwestie użycia wskaźników w kodzie wielowątkowym:
\begin{itemize}
\item Co stanie się po przekazaniu tego samego \code{MessagePtr} do 2 różnych instancji \code{MsgQueue}? czy będzie prawidłowo usunięty? Czy występuje ryzyko podwójnego usunięcia? Czy zliczanie referencji w \code{shared\_ptr} jest bezpieczne? czy jest to wydajne?
\item Co stanie się gdy kilka wątków zacznie modyfikować wiadomość udostępnianą przez \code{MsgQueue}?
\item Czy podobne kwestie są aktualne dla \code{weak\_ptr} i \code{unique\_ptr}
\end{itemize}

\paragraph{podsumowanie problemu}
W dostępie do zasobów należy uwzględnić następujące kwestie:
\begin{itemize}
\item Funkcje biblioteczne i systemowe niewspierające wielowątkowości.
\item Inicjację zasobu.
\item Odległość zasobu i blokady.
\item Wskaźniki i referencje.
\item Głębokość sekcji krytycznych.
\item Atomowość sekcji krytycznych.
\item Zachowanie spójność zasobu między sekcjami krytycznymi.
\item Unikanie wywołania zewnętrznego kodu wewnątrz sekcji krytycznych.
\end{itemize}

%\section{Kwestie otwarte}\label{sec:open-cases}

\paragraph{Wyścig interface'ów}\label{par:interface-race}
Wrapper \code{Resource} rozwiązuje część najprostszych problemów. Jak jednak rozwiązać możliwy wyścig interface'ów?
Należy tak projektować interface korzystający z zasobów by ten działał w formie bezstanowej. Jak to rozumieć?
Kod klienta zasobu nie powinien zakładać stanu obiektu z wywołania poprzedniej sekcji krytycznej. Wyobraźmy sobie następujący scenariusz dla \code{MsgHandler} dla:
\\msg1 = [prio:0,source:"a",text:"txt1"],
\\msg2 = [prio:1,source:"b",text:"txt2"],
\\msg3 = [prio:4,source:"a",text:"txt3"]
\\\inlineCpp{(t:1) s\_handler.set(msg1)}
\\\inlineCpp{(t:1) s\_handler.set(msg2)}
\\\inlineCpp{(t:1) auto msg = s\_handler.top()}
\\\inlineCpp{(t:2) s\_handler.set(msg3)}
\\\inlineCpp{(t:1) s\_handler.remove(msg)}
\\W wyniku nigdy nie odczytamy wiadomości "txt3".

Jak sprawić by t:2 nie weszło między wywołania t:1? Mam na to 2 rady:
\begin{itemize}
\item Najlepiej przeprojektować interface tak aby wszystkie jego operacje były atomowe w tym operacje wykonujące wiecej niż 1 czynność, jak pobranie z jednoczesnym usunięciem elementu.
\item Alternatywnie można Wyprowadzić blokowanie poza klasę \code{MsgHandler} - np. przez wyposażenie jej w publiczną metodę blokującą zwracającą obiekt reprezentujący blokadę.
\end{itemize}

<<Dać przykład dla obu rozwiązań>>

\paragraph{Wskaźniki do zasobów}
Obiekty klasy \code{Message} są przekazywane poprzez \code{shared\_ptr} w obrębie całego programu. Rodzi to pewne korzyści ale też problemy związane z naturą tego wskaźnika.

Umieszczenie obiektu we współdzielnej pamięci wolnej automatycznie nadaje mu tożsamość wyrażoną przez adres pamięci którą zajmuje. Przenoszenie obiektu z pamięci wolnej między wątkami jest bardzo tanie i często jest operacją atomową. Tożsamość obiektu jest bardzo przydatna gdy chcemy przeciwdziałać wyścigom interface'u - dla przykładu w paragrafie \ref{par:interface-race} omówiono wyścig który możnabyłoby rozwiązać gdyby \code{remove()} porównywał adres (tożsamość) podanej w argumencie (a wcześniej zwróconej przez \code{top()}) z wiadomościami obecnymi w kolejce. Korzystanie z \code{shared\_ptr} w kodzie wielowątkowym jest bezpieczne pod względem zliczania odniesień i destrukcji przechowywanego obiektu.

Negatywnymi skutkami użycia \code{shared\_ptr} są natomiast większe skomplikowanie kodu, brak jednoznacznie określonego własciciela obiektu, odwołania cykliczne i wycieki pamięci. Nieuważne operowanie wskaźnikami może prowadzić do wyścigów danych.
Alternatywą dla \code{shared\_ptr} jest \code{unique\_ptr}, który posiada wszystkie zalety wskaźników, bez jednoczesnych problemów związanych z jego współdzieleniem - odpowiada za poprawną destrukcję obiektu jednocześnie gwarantuje że jest jego jedynym właścicielem. Jest to preferowany wskaźnik w kodzie wielowątkowym.

W kodzie przyjęte jest że jedynym właścicielem jest \code{unique\_ptr}, natomiast gołe wskaźniki pobierane za pomocą \code{unique\_ptr.get()} służą tylko odnoszeniu się do obiektu.
Podobna - bardziej sformalizowana relacja zachodzi między \code{shared\_ptr} a \code{weak\_ptr}, gdzie \code{weak\_ptr} śledzi obiekt przechowywany przez \code{shared\_ptr}.
\code{weak\_ptr} tak jak gołe wskaźniki nie gwarantują istnienia wskazanego przez nie obiektu, z tą różnicą że \code{weak\_ptr} pozwala sprawdzić czy obiekt jeszcze istnieje.

\paragraph{Funcje systemowe}
Standard \Cpp{}20 daje wsparcie dla współbieżnych operacji wejścia/wyjścia za pomocą \code{<syncstream>}. Co robić gdy mamy wielokrotne odwołania do \code{std::cout} w całym programie? Można użyć jednej z 2 metod:
\begin{itemize}
\item przekazać cały string - operator <{}< jest threadsafe - czyli jeśli przekażemy argument w całości zostanie wyświetlony w całości. Można wspomagać się lokalnym buforem <<przykładowy kod>>
\item użyć \code{ios\_base::register\_callback()} do synchronizacji i zablokowania muteksu. Działa także dla bibliotek zależnych.
\end{itemize}

\paragraph{Głębokie sekcje krytyczne}
Jak zablokować głębokie sekcje krytyczne? W sekcji krytycznej można zastosować asercję sprawdzającą czy nie jesteśmy już w innej sekcji krytycznej \code{Resource<>}. Można to łatwo zaimplementować z pomocą lokalnych zmiennych statycznych. <<przykładowy kod>>

\paragraph{Widoczność sekcji krytycznych}
Jedną z rzeczy które możnaby poprawić w klasie Resource jest widoczność. \inlineCpp{getGlobalCfg().critical\_section(...)} nie wyróżnia się w kodzie. Można poprawić to na 2 sposoby:
\begin{itemize}
\item makro\\
  \inlineCpp{CRITICAL\_SECTION(getGlobalCfg(),\{ ... \})}
\item klasa\\
  \inlineCpp{ResourceRef<CfgReader> res = getGlobalCfg();}\\
  \inlineCpp{res.critical\_section([](auto const\& x)\{...\})}
\end{itemize}

\theendnotes
\end{document}
