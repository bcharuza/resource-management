\documentclass[titlepage,twocolumn,a4paper,10pt,fleqn,leqno]{article}

\usepackage[utf8]{inputenc}
\usepackage[T1]{fontenc}
\usepackage{polski}
\usepackage{titlesec}
\usepackage{wrapfig}
\usepackage{listings}
\usepackage{blindtext}
\usepackage{xcolor}

\definecolor{red}{rgb}{1,0,0}
\definecolor{codegreen}{rgb}{0,0.6,0}
\definecolor{codegray}{rgb}{0.5,0.5,0.5}
\definecolor{codepurple}{rgb}{0.58,0,0.82}
\definecolor{backcolour}{rgb}{0.95,0.95,0.95}

\lstdefinestyle{cpp-style}{
    backgroundcolor=\color{backcolour},
    commentstyle=\color{codegreen},
    keywordstyle=\color{magenta},
    numberstyle=\tiny\color{codegray},
    stringstyle=\color{codepurple},
    basicstyle=\ttfamily\scriptsize,
    breakatwhitespace=false,
    breaklines=true,
    captionpos=b,
    keepspaces=false,
    numbers=left,
    numbersep=1pt,
    showspaces=false,
    showstringspaces=false,
    showtabs=false,
    tabsize=2
}

\newcommand\includeCpp[2]{\noindent\lstinputlisting[language=C++,style=cpp-style,captionpos=t,title=\scriptsize#2]{#1/#2}}
\newcommand\includeText[2]{\noindent\lstinputlisting[captionpos=t,style=cpp-style,title=\scriptsize#2]{#1/#2}}

\newcommand\inlineBash[1]{\colorbox{backcolour}{\lstinline[language=bash,style=cpp-style,frame=lines]{$> #1}}}

\newcommand\inlineCpp[1]{\colorbox{backcolour}{\lstinline[language=C++,style=cpp-style]{#1}}}

\titleformat{\section}
  {\normalfont\large\bfseries}{\thesection.}{1em}{}
\titleformat{\subsection}
  {\normalfont\normalsize\itshape}{\thesubsection.}{1em}{}
\titleformat{\subsubsection}
            {\normalfont\normalsize\itshape}{\thesubsubsection.}{1em}{}

\usepackage{relsize,endnotes,xurl,hyperref}

\renewcommand\UrlFont{\rmfamily}

\newcommand{\note}[2]{#1\endnote{#2}}

\newcommand{\link}[1]{\url{#1}}

\newcommand{\mail}[1]{\href{mailto:#1}{#1}}

\newcommand{\Rsign}[1]{\protect\hspace{-.15em}\protect\raisebox{.4ex}{\smaller\smaller\smaller\textbf{#1}}}
\newcommand{\Cpp}[1]{\mbox{C\Rsign{+}\Rsign{+}\protect\hspace{-.15em}#1}}

\makeatletter
\renewcommand\paragraph{%
  \@startsection{paragraph}{4}{0mm}%
                {-\baselineskip}%
                {0.5\baselineskip plus 0.2\baselineskip minus 0.2\baselineskip}%
                {\normalfont\normalsize\bfseries}}
\makeatother

\newcommand\editNote[2]{#1\textit{\color{red}{<<#2>>}}}

\newcommand\code[1]{\texttt{#1}}
\newcommand\cname[1]{\code{#1}}
\newcommand\fname[1]{\code{#1()}}
\newcommand\tname[1]{\code{#1<>}}

\include{lib/title-page}

\setTitle{Współdzielenie zasobów}
\setTitleCatchphrases{\Cpp, Przetwarzanie współbieżne, Struktura programu}
\setTitlePhoto{images/pizza}
\setTitleDescription{Artykuł prezentuje wybrane metody dostępu do zasobów w przykładach wzorowanych na kodzie produkcyjnym. Autor omawia problematykę, a następnie wskazuje metody usystematyzowania dostępu do elementów współdzielonych.}

\setAuthor {Bartosz Charuza}
\authorPhoto {images/author}
\authorPosition {Regular \Cpp{} Software Developer}
\authorProject {Alstom: Ebiscreen2000}
\authorWebpage {\link{https://github.com/bcharuza}}
\authorEmail {bcharuza@yahoo.com}

\hypersetup{
  pdfauthor=\getAuthor,
  pdftitle=\getTitle,
  pdfpagemode=None,
  urlbordercolor=black,
  pdfborderstyle={/S/U/W 0.3}}


\begin{document}
\begin{titlepage}
\maketitle
\end{titlepage}
\section{Wstęp}\label{sec:introduction}
\paragraph{Dlaczego napisałem ten artykuł?}
Nie lubię debugować i debuggerów w ogóle. Nie znam gorszego losu, niż debugowanie aplikacji wielowątkowych. Gdy siedzę przed debuggerem czuję się jak kot goniący laserową kropkę drwiącego ze mnie absolutu - z tym że kot dobrze się przy tym bawi.
Pewnego pięknego październikowego ranka dostałem zwrot ,,XYZ czasami nie działa'' to oznaczało, że czeka mnie kilka dni przedniej zabawy - moje autystyczne rand ez-vous z debuggerem i tysiącami linii kodu. Błędem okazało się użycie niewłaściwego muteksu - Straciłem 2 dni życia na prosty błąd. To był punkt przełomowy. Cisnąłem ,,Tak być nie może!'' i przepisałem tę bibliotekę. \note{Frustracja to potężna muza - przykładowo niedziałająca drukarka zmieniła losy świata.}{\link{https://en.wikipedia.org/wiki/Richard\_Stallman\#Events\_leading\_to\_GNU}}

\paragraph{Kto jest adresatem?}
Programiści \Cpp{}, od juniora który może znaleźć inspirację, po seniora który może chcieć wymagać podobnych rozwiązań w swoim projekcie. Zakładam że czytelnik posiada podstawową znajomość \Cpp{14} i wiedzę z zakresu przetwarzania wielowątkowego z użyciem mutexów biblioteki standardowej.

\paragraph{Jaki jest zakres artykułu?}
Publikacja prezentuje wzorzec \tname{Resource} i przypadki użycia. Omawiane będą kontrprzykłady, zagadnienia inicjacji i dostępu do zasobów. Nie opisuję narzędzi, ani podstaw wielowątkowości. Większość poruszanych tematów będzie związana z wykorzystaniem wspomnianego wzorcu.

Prezentowane metody sprawdzone są w aplikacjach klasy desktop przy użyciu mechanizmów biblioteki standardowej do kilkudziesięciu wątków. Nie będą poruszane rozwiązania dla aplikacji serwerowych, czy o wysokich wymaganiach wydajnościowych w których nie mam doświadczenia. Artykuł nie ma też zastosowania dla problematyki massive-parallelizm.

\paragraph{Struktura artykułu}
Próbowałem napisać ten artykuł w standardowej akademickiej strukturze problem-analiza-rozwiązanie. Skutkowało to zbyt rozciągłą formą wraz dojściem nowych wątków, dywagacji i przypadków szczególnych.

Zdecydowałem się zastosować formułę rozwiązanie-zastosowanie, gdzie przypadki użycia są podzielone na krótkie pary kontrprzykład-przykład. Mam nadzieję że pozwoli to zredukować rozmiar do minimum i dotrwać czytelnikowi do końca artykułu.

\section{Przygotowanie merytoryczne}\label{sec:reader-profile}
\paragraph{Zasób}
\note{Zasób}{\link{https://en.wikipedia.org/wiki/System\_resource}} to coś o ograniczonej dostępności. Zasoby w przeciwieństwie do informacji/wartości posiadają tożsamość. Zasobem może być urządzenie zewnętrzne, pamięć, czas procesora, a nawet obiekt.

\paragraph{Typ}
\note{Typ określa zestaw dozwolonych wartości i operacji obiektu.}{Bjarne Stroustrup: ,,Język \Cpp. Kompendium wiedzy.''}

\paragraph{Wartość}
\note{Wartość to zbiór bitów interpretowanych zgodnie z typem}{Bjarne Stroustrup: ,,Język \Cpp. Kompendium wiedzy.''}

\paragraph{Obiekt}
\note{Obiekt to miejsce w pamięci, w którym przechowywana jest wartość jakiegoś typu}{Bjarne Stroustrup: ,,Język \Cpp. Kompendium wiedzy.''}

\paragraph{Znajomość \Cpp{}14}
Większość kodu będzie napisana w \Cpp{}14. Czytelnik znający \Cpp{}11 nie powinien widzieć różnicy. Czytelnik znający \Cpp{}03 powinien być w stanie zrozumieć treść. Starsze standardy różnią się zbyt mocno aby artykuł mógł być łatwo przyswajalny.
\begin{itemize}
\item \link{https://en.cppreference.com/}
\item Bruce Eckel: ,,Thinking in \Cpp''
\item Bjarne Stroustrup: ,,Język \Cpp. Kompendium wiedzy.''
\end{itemize}

\paragraph{Przetwarzanie współbieżne}
Znajomość mutexów i podstaw wielowątkowości biblioteki standardowej \Cpp{} powinna być wystarczająca.
\begin{itemize}
\item Anthony Williams ,,Język \Cpp{} i przetwarzanie współbieżne w akcji''
\end{itemize}

\paragraph{Kompilacja i linkowanie}
By zrozumieć niektóre problemy czytelnik powinien umieć kompilować pliki źródłowe do plików binarnych a następnie potrafić je linkować w 2 osobnych krokach.

\paragraph{Programowanie szablonowe}
Część kodu jest kodem szablonowym. Do pełnego zorumienia implementacji należy poznać wykorzystanie \code{std::enable\_if<>} i pozostałych narzędzi programowania szablonowego udostępnianego przez bibliotekę standardową.

\section{Środowisko testowe}
\begin{tabular}{l r}
  Kompilator:   & GCC v9.3\\
  System:       & Ubuntu 20.04.2 LTS\\
  Architektura: & x86\_64
\end{tabular}

\section{Abstrakcja zasobu}\label{sec:resource}
Po wielokrotnym napotkaniu przedstawionego wcześniej kodu, zacząłem myśleć jak uprościć życie sobie, i towarzyszom niedoli. Jak sprawić by użytkownik kodu nie musiał zastanawiać się jak, gdzie i po co blokować zasób i pogramować szęśliwie jak wcześniej w stylu jednowątkowym bez trosk i zmartwień wielowątkowej dżungli.

\paragraph{\code{Resource<T>}}
W poprzednim kodzie błędnie ustalano zakresy muteksów. Zacząłem więc od silnego związania zasobu z zewnętrznym dla niego muteksem. Tak zaprojektowany wrapper sam ściągałby z użytkownika blokowanie i zwalnianie muteksu. Dobrze byłoby też jasno zaznaczyć obszar sekcji krytycznej, by użytkownik szablonu wiedział gdzie musi się pilnować. Aby zagwarantować zwalnianie muteksu po wyrzuceniu wyjątku użyto \code{lock\_guard<>}. 
\includeCpp{../../examples/resource}{Resource.hpp}
Klasa jest prosta i skupia się wyłącznie na zajmowaniu zasobu. Inicjacja wrapperu odbywa się transparentnie dla obiektu i nie zakłada dodatkowych zabezpieczeń - wtórca obiektu musi dopilnować, że inicjacja nie będzie przebiegała na globalnej stercie. Dostęp do chronionego obiektu odbywa się za pomocą wizytatora poprzez metodę \code{critical\_section()}.

\paragraph{\code{

\section{Przykłady}\label{sec:examples}
Klasa \tname{Resource} pomogła mi w zwalczaniu błędów występujących w poprawianym kodzie:
\begin{itemize}
\item brak pokrycia zakresów muteksu i zasobu.
\item długie sekcje krytyczne.
\item jednoczesne zajmowanie wielu zasobów -- w tym obsługa operacji I/O.
\item wyścigi interface'u.
\end{itemize}

\paragraph{CfgReader}
Pierwszym przykładem, który chcę omówić jest klasa \cname{CfgReader} -- silnie zredukowany przedstawiciel kodu który spotkałem w produkcji. Działanie jest banalne - po wczytaniu konfiguracji za pomocą metody \fname{LoadCfg} aktualizowane są pola statyczne klasy.
\includeCpp{../../examples}{CfgReader.hpp}
\includeCpp{../../examples}{CfgReader.cpp}
Można zauważyć następujące błędy:
\begin{itemize}
\item \fname{LoadCfg} ma zakres instancji a modyfikowane pola są statyczne - to mylące i niebezpieczne.
\item Nie da się określić czy pola zostały zainicjowane czy jeszcze nie.
\item Przeładowanie konfiguracji nie jest operacją atomową.
\end{itemize}
Powyższy kod działał tylko dlatego, że w całym programie występowała tylko jedna instancja, której metoda była wywoływana we wczesnej fazie uruchamiania programu, a cała reszta programu starała się nie korzystać z konfiguracji przed jej załadowaniem -- tylko kto zagwarantuje że faktycznie tak faktycznie robi.

Wymusza to na konfiguracji by była inicjowana jak najwcześniej podczas rozruchu aplikacji i by była niezmienia w dalszym ciągu działania programu. Korzystanie z pól statycznych klasy przed wywołaniem \fname{LoadCfg} musi być pilnowane przez programistę. Po inicjacji konfiguracja nie może być już przeładowana, i jest sprowadzona do zbioru wartości. Nie jesteśmy nawet w stanie kontrolować przypadkowej modyfikacji.

Jak użycie \tname{Resource} jest w stanie poprawić kod naszej klasy? 

Załóżmy że jesteśmy leniwymi ludźmi, lub ujmując zgrabniej - minimalistami. Jak poprawić nasz kod niewielkim wysiłkiem?
\includeCpp{../../examples}{CfgReader-fix.hpp}
\includeCpp{../../examples}{CfgReader-fix.cpp}
Co zmieniłem?
Przeniosłem zarządzanie zasobem na wrapper \tname{Resource} i udostępniam przez funkcję globalną i nic więcej mnie nie interesuje. Dostęp jest kontrolowany wewnętrznie. Mapa jest zajmowana w miejscu użycia, aktualizowana w pojedynczym przypisaniu i momentalnie zwalniana dla reszty wątków.

Dlaczego użyłem funkcji dostępowej zamiast umieścić zasób w zmiennej globalnej?
Ponieważ standard definiuje kolejność inicjacji zmiennej globalnej w zależności od pierwszeństwa jej wystąpienia, co oznacza że jest ona zależna od kolejności kompilacji/linkowania. Aby to objeść użyłem prostego rozwiązania - funkcji globalnej której pola statyczne muszą być zainicjowane przed jej wywołaniem. Kompilator automatycznie ustali kolejność inicjacji. Dodatkowo pozwala to na prostszą refaktoryzację i testowanie kodu.

\paragraph{Logger}
\cname{Logger} reprezentuje typowy kod C przełożony na grunt \Cpp{}. Jest to silnie zredukowany przykład wzorowany na popularnych leniwych logerach, często przeplatanych niezależnymi wywołaniami do funkcji wyściowych. Jest to prawdziwy kłopot dla kodu wielowątkowego.
\includeCpp{../../examples}{Logger.hpp}
\includeCpp{../../examples}{Logger.cpp}
W powyższym kodzie są wykorzystywane funkcje systemowe które nie dają gwarancji bezpieczeństwa lub gwarancja jest zależna od implementacji biblioteki jak \fname{vfprintf}. \note{W systemach zgodnych z POSIX funkcje rodziny printf są bezpieczne, jednak ctime już nie musi}{\link{https://pubs.opengroup.org/onlinepubs/9699919799/functions/V2\_chap02.html}}. Dla Windows nie spotkałem się z taką gwarancją. Używając bibilioteki C trzeba szczególnie zwracać uwagę czy wybrany element jest przystosowany do użycia w środowisku wielowątkowym. Dostęp do konfiguracji jest niechroniony i może powodować liczne błędy wyścigu do danych.

Skoro C nie daje gwarancji dla wielowątkowego dostępu do wejścia/wyjścia to może \Cpp{}?
Dla porównania wg. \note{propozycji p0053r7}{\link{https://isocpp.org/files/papers/p0053r7.pdf}} standard zapewnia bezpieczeństwo dla strumieni wyjściowych takich jak \vname{std::cout}. Bezpieczeństwo jest gwarantowane tylko dla pojedynczej operacji, więc \code{cout<<A<<B} może już zostać przerwane przez inny proces między wywołaniem \code{<<A} i \code{<<B}. Należy też uważać na starsze wersje biblioteki \lib{stdc++}, która mogła niepoprawnie obsługiwać wielowątkowe operacje na obiektach strumieni.

Spróbujmy przeprojektować powyższy kod, tak by zachować wygodę \fname{vfprintf} i zagwarantować bezpieczeństwo wielowątkowe.

\includeCpp{../../examples}{Logger-fix.hpp}
\includeCpp{../../examples}{Logger-fix.cpp}
Co zmieniono?
Funkcję \fname{\tname{trace}} kompletnie przepisano, tak by używała tylko biblioteki \Cpp{}. Wyświetlenie wiadomości odbywa się w pojedynczym wywołaniu \fname{std::ostream::operator <<} - mamy więc pewność że kod jest bezpieczny ze względu na wątki. Zapełnianie buforu oss odbywa się w pamięci lokalnej więc wszelkie interakcje z zasobem \vname{std::cout} są ograniczone do minimum.

Funkcję \fname{trace\_sysinfo} jest odpowiedzialna za zapisanie nagłówku wiadomości i przeniesiono ją do pliku .cpp by ograniczyć zależności nagłówka.

Kod \fname{\tname{trace\_impl}} przedstawia rekurencyjne wywołania szablonowe o zmiennej liczbie argumentów. Jest to rozwiąznie dające możliwości variadic functions znanych z C i jednocześnie zachowujących ścisłą kontrolę typów znaną z \Cpp{}. Dodatkowo użytkownik może definiować własne przeciążenia \fname{\tname{trace\_print}} dzięki czemu w prosty sposób można dodać obsługę typów złożonych. Funkcje pomocnicze umieszczono w tzw. anonimowej przestrzeni nazw, by uniemożliwić dostęp do nich spoza pliku.

Warto zauważyć że dostęp do konfiguracji odbywa się przez \fname{getCfg} atomowo w krótkiej sekcji krytycznej. Obsługa sekcji krytycznej jest niemal niezauważalna dla użytkownika kodu i objawia się użyciem lambdy.

\paragraph{MessageHandler}
Poniżej przedstawię klasę \cname{MessageHandler} która jest inspirowana klasą z kodu produkcyjnego i przedstawia częste bolączki niewłaściwie zaprojektowanej sieci powiązań.
\includeCpp{../../examples}{MessageHandler.hpp}
\includeCpp{../../examples}{MessageHandler.cpp}
W powyższym kodzie klasa \cname{MsgHandler} pełni funkcję centralnej koordynacji przepływu wiadomości. Zawiera jeden globalny interface we/wy \cname{MQClient} z którym związuje się przez wymianę adresów. Już na tym etapie widać pierwszą pułapkę - adres nic nie mówi o czasie życia obiektu i gdy jeden z obiektów przestanie istnieć narażamy się na naruszenie ochrony pamięci. Drugim trudniej zauważalnym problemem jest pełzająca spagettyfikacja kodu - tworzenie takich odniesień 1-1 powoduje że adres zasobu będzie przekazywany między instancjami klienckimi. Modyfikacja i zrozumienie zależności takiego kodu będą z czasem coraz trudniejsze. Aby tego uniknąć zasoby powinny być umieszczane w globalnych repozytoriach.

Co gorsze połączenia wcale nie muszą być symetryczne. Po wielokrotnym wywołaniu \fname{setClient} dla różnych instancji \cname{MsgHandler} otrzymamy wiele instancji \cname{MQClient} wskazujących na różne \cname{MsgHandler}, które z kolei wskazują na ostatnią powiązaną instancję \cname{MQClient}. To nie może skończyć się dobrze.

Poza we/wy klasa \cname{MsgHandler} zarządza także mechanizmem subskrybcji które polegają na wywołaniu callbacku po otrzymaniu nowej wiadomości. Subskrybcje są współdzielone między instancjami \cname{MsgHandler}, a subskrybenci są informowani sekwencyjnie wewnątrz metody \fname{NotifyAll}, która też jest daleka od idału. Notyfikacje wymagają sekwencyjnego wykonywania podanych wywołań zwrotnych w tym samym wątku co kod dodający wiadomość do kolejki, a zablokowanie jednego z nich doprowadzi do niepowiadomienia kolejnych subskrybentów. Pomojając możliwe zakleszczenia powodowane wywołaniem kodu klienckiego pewne są też długie czasy wykonania w wątku wywołującym. Przeniesienie obsługi notyfikacji do osobnego wątku mogłoby poprawić responsywność, jednak jest to obecnie spychane na użytkownika klasy.

Ostatnim z obowiązków klasy \cname{MsgHandler} jest zarządzanie kolejką wiadomości która także dzielona między instancjami. Sama klasa \cname{MsgQueue} jest zrealizowana prawidłowo, jednak nie jest ona przystosowana do kodu wielowątkowego. Nie jest to problemem gdyż jest chroniona przez \cname{MsgHandler}, która jednak zezwala na tzw. wyścig interface'u. Wyścig interface'u polega na tym że klasa zakłada niejawną transakcję - tj. że klient musi pamiętać jaki jest stan zasobu między 2 kolejnymi operacjami który może być w międzyczasie zmodyfikowany przez inny wątek. Przykładowo mając 2 wątki \code{A} i \code{B} i współdzielony obiekt \code{MsgHandler x}:

\begin{tabular}{l | l}
  \code{A} & \code{B}\\
  \hline
  \code{if(x.isEmpty())} & \code{if(x.isEmpty())}\\
  \code{  x.sendTop()} & \code{  x.sendTop()}
\end{tabular}

\noindent
w przypadku gdy w \vname{x} jest tylko 1 element, w zależności od kolejności wykonania instrukcji powyższy kod może być wykonany poprawnie, lub doprowadzić do załamania aplikacji, pomimo że zarówno \fname{isEmpty} jak i \fname{sendTop} są wykonywane jako sekcje krytyczne.

Kończąc, pozostaje zauważyć że jak na ironię jedynym elementem który nie jest dzielony między instancjami jest muteks który powinien chronić elementy współdzielone przed jednoczesnym dostępem. Obecnie nie spełnia on swojej roli.

Klasa \cname{MsgHandler} będzie działała poprawnie tylko w środowisku jednowątkowym, jednak gdy dla danego obiektu wystąpią równoczesne odwołaniado \fname{send} obiekt klasy zacznie zachowywać się w sposób niezdefiniowany. Odbiór wiadomości jest przedstawiony za pomocą metody \fname{receive}.

Strukturę tego programu można szybko poprawić za pomocą globalnych repozytoriów zasobów i wrapperu \tname{Resource}. Pytaniem otwartym jest - co potraktujemy jako zasób? Aby to zrobić trzeba przeanalizować przypadki użycia i przepływ informacji. Wiemy że:
\begin{itemize}
\item jedyną klasą posiadającą dane współdzielone jest \cname{MsgHandler}.
\item Każda z instancji \cname{MQClient} może komunikować się z jedną instancją \cname{MsgHandler}
\item Każda z instancji \cname{MsgHandler} musi komunikować się z ostatnią skonfigurowanę \cname{MQClient}.
\end{itemize}

Wyłania się z tego obraz systemu muxującego (nie mylić z mutexem) z pamięcią. System muksujący kieruje wiele sygnałów wejściowych na jedno wyjście. Obec tego powinniśmy posiadać możliwość tworzenia nieograniczonej ilości instancji \cname{MQClient} i posiadać możliwość zarejestrowania 1 instancji jako wyjścia \cname{MsgHandler}. Moim zdaniem taki hub nie powinien móc współdzielić pamięci między swoimi instancjami więc subskrybcje i kolejka powinny być przechowywane lokalnie. Metody \cname{MsgHandler} będą wykonywane współbieżnie, więc instancje tej klasy traktować należy jako zasób.
\includeCpp{../../examples}{MessageHandler-fix.hpp}
\includeCpp{../../examples}{MessageHandler-fix.cpp}
Co się zmieniło? Najważniejszą zmianą jest dodanie \fname{initializeMsgHandler} i \fname{accessMsgHandler} które przyjmują formę prostego repozytorium dla obiektów \cname{MsgHandler}.

W \cname{MQClient} nie zmieniło się nic poza sposobem przechowywania odwołania muxu z prostego wskaźniku na \tname{shared\_ptr}. Pobranie instancji \cname{MsgHandler} odbywa się tylko z repozytorium zasobów. Warto nadmienić, żę wykorzystano tu bezpieczeństwo zliczania odniesień \tname{shared\_ptr} dla wywołań współbieżnych. Dostępem współbieznym administruje zaś wrapper \tname{Resource}. Alternatywnym rozwiązaniem może być też przechowywanie nazwy/id instancji muxu i każdorazowe pobieranie instancji z repozytorium.

\cname{MsgHandler} został mocno zmieniony. Wchłonięto klasę \cname{Subscription} która wyłącznie obciążała użytkownika - teraz operacje nadawania unikalnego ID odbywają się we wnętrzu klasy w \fname{getNewId}. Wszystkie pola klasy są polami instancji dzięki czemu unikamy konieczności pilnowania dostępu do nich.

Z \cname{MsgHandler} usunięto muteks instancji. Nie jest to rozwiązanie czysto pozytywne - ma wady i zalety. Wadą jest porzucenie kontroli nad dostępem do klasy - przez co dla środowiska wielowątkowego musi być zawsze otoczona przez wrapper \tname{Resource}. Niewątpliwą zaletą jest jednak brak konieczności zajmowania się dostępem wielowątkowym. Usunięto mutex, którego nie trzeba pilnować w kodzie i prawdopodobieństwo wystąpienia wyścigu interface'u zostało znacznie zredukowane.

%\section{Opis problemu}
Za punkt wyjścia przyjmiemy spreparowany kod złożony z 3 par plików \code{Logger.hpp/.cpp}, \code{MessageQueue.hpp/.cpp}, \code{CfgReader.hpp/.cpp} i pliku głównego \code{main.cpp}.
Przykłady są syntezą kodu produkcyjnego zastanego w kilku projektach z którymi pracowałem. 

\paragraph{Logger}
\includeCpp{../../examples/problem}{Logger.hpp}
\includeCpp{../../examples/problem}{Logger.cpp}
Powyższy przykład reprezentuje typowy kod C przełożony na grunt \Cpp{}. Często używane są bezpośrednio funkcje systemowe które nie dają gwarancji bezpieczeństwa lub gwarancja jest zależna od implementacji biblioteki jak \code{vfprintf}. W systemach zgodnych z POSIX funkcje printf są bezpieczne \link{https://pubs.opengroup.org/onlinepubs/9699919799/functions/V2\_chap02.html\#tag\_15\_09}, jednak dla Windows nie spotkałem się z taką gwarancją.

Dla porównania wg. \link{https://isocpp.org/files/papers/p0053r7.pdf} \code{std::cout} zapewnia bezpieczeństwo. Jest to jednak złudne gdyż z moich doświadczeń jest ono zapewnione tylko w obrębie pojedynczej operacji - \code{cout<<A<<B} może już zostać przerwane przez inny proces.

Dostęp do statycznego \code{s\_category} jest potencjalnie bezpieczny (nie licząc zbyt dużego indeksu). Przyczyną jest gwarantowane przez język automatyczne czyszczenie przechowywanych stringów po wyjściu z programu, i dostęp wyłącznie do odczytu - przez co obiekt posiada charakterystykę bardziej wartości niż obiektu. Problem dostępu do \code{s\_category} może wystąpić w tylko jednym przypadku - jeśli użytkownikowi uda się odwołać do obiektu jeszcze przed jego inicjacją - tzn. gdy \code{trace()} zostanie wywołany w kodzie inicjacyjnym w innej jednostce kompilacji, przed ustawieniem wartości \code{s\_category}.
\paragraph{CfgReader}
\includeCpp{../../examples/problem}{CfgReader.hpp}
\includeCpp{../../examples/problem}{CfgReader.cpp}
Reader opis

\paragraph{MessageQueue}
\includeCpp{../../examples/problem}{MessageQueue.hpp}
\includeCpp{../../examples/problem}{MessageQueue.cpp}
MessageQueue opis

\paragraph{main}
\includeCpp{../../examples/problem}{main.cpp}

%\section{Kwestie otwarte}\label{sec:open-cases}

\paragraph{Wyścig interface'ów}\label{par:interface-race}
Wrapper \code{Resource} rozwiązuje część najprostszych problemów. Jak jednak rozwiązać możliwy wyścig interface'ów?
Należy tak projektować interface korzystający z zasobów by ten działał w formie bezstanowej. Jak to rozumieć?
Kod klienta zasobu nie powinien zakładać stanu obiektu z wywołania poprzedniej sekcji krytycznej. Wyobraźmy sobie następujący scenariusz dla \code{MsgHandler} dla:
\\msg1 = [prio:0,source:"a",text:"txt1"],
\\msg2 = [prio:1,source:"b",text:"txt2"],
\\msg3 = [prio:4,source:"a",text:"txt3"]
\\\inlineCpp{(t:1) s\_handler.set(msg1)}
\\\inlineCpp{(t:1) s\_handler.set(msg2)}
\\\inlineCpp{(t:1) auto msg = s\_handler.top()}
\\\inlineCpp{(t:2) s\_handler.set(msg3)}
\\\inlineCpp{(t:1) s\_handler.remove(msg)}
\\W wyniku nigdy nie odczytamy wiadomości "txt3".

Jak sprawić by t:2 nie weszło między wywołania t:1? Mam na to 2 rady:
\begin{itemize}
\item Najlepiej przeprojektować interface tak aby wszystkie jego operacje były atomowe w tym operacje wykonujące wiecej niż 1 czynność, jak pobranie z jednoczesnym usunięciem elementu.
\item Alternatywnie można Wyprowadzić blokowanie poza klasę \code{MsgHandler} - np. przez wyposażenie jej w publiczną metodę blokującą zwracającą obiekt reprezentujący blokadę.
\end{itemize}

<<Dać przykład dla obu rozwiązań>>

\paragraph{Wskaźniki do zasobów}
Obiekty klasy \code{Message} są przekazywane poprzez \code{shared\_ptr} w obrębie całego programu. Rodzi to pewne korzyści ale też problemy związane z naturą tego wskaźnika.

Umieszczenie obiektu we współdzielnej pamięci wolnej automatycznie nadaje mu tożsamość wyrażoną przez adres pamięci którą zajmuje. Przenoszenie obiektu z pamięci wolnej między wątkami jest bardzo tanie i często jest operacją atomową. Tożsamość obiektu jest bardzo przydatna gdy chcemy przeciwdziałać wyścigom interface'u - dla przykładu w paragrafie \ref{par:interface-race} omówiono wyścig który możnabyłoby rozwiązać gdyby \code{remove()} porównywał adres (tożsamość) podanej w argumencie (a wcześniej zwróconej przez \code{top()}) z wiadomościami obecnymi w kolejce. Korzystanie z \code{shared\_ptr} w kodzie wielowątkowym jest bezpieczne pod względem zliczania odniesień i destrukcji przechowywanego obiektu.

Negatywnymi skutkami użycia \code{shared\_ptr} są natomiast większe skomplikowanie kodu, brak jednoznacznie określonego własciciela obiektu, odwołania cykliczne i wycieki pamięci. Nieuważne operowanie wskaźnikami może prowadzić do wyścigów danych.
Alternatywą dla \code{shared\_ptr} jest \code{unique\_ptr}, który posiada wszystkie zalety wskaźników, bez jednoczesnych problemów związanych z jego współdzieleniem - odpowiada za poprawną destrukcję obiektu jednocześnie gwarantuje że jest jego jedynym właścicielem. Jest to preferowany wskaźnik w kodzie wielowątkowym.

W kodzie przyjęte jest że jedynym właścicielem jest \code{unique\_ptr}, natomiast gołe wskaźniki pobierane za pomocą \code{unique\_ptr.get()} służą tylko odnoszeniu się do obiektu.
Podobna - bardziej sformalizowana relacja zachodzi między \code{shared\_ptr} a \code{weak\_ptr}, gdzie \code{weak\_ptr} śledzi obiekt przechowywany przez \code{shared\_ptr}.
\code{weak\_ptr} tak jak gołe wskaźniki nie gwarantują istnienia wskazanego przez nie obiektu, z tą różnicą że \code{weak\_ptr} pozwala sprawdzić czy obiekt jeszcze istnieje.

\paragraph{Funcje systemowe}
Standard \Cpp{20} daje wsparcie dla współbieżnych operacji wejścia/wyjścia za pomocą \code{<syncstream>}. Co robić gdy mamy wielokrotne odwołania do \code{std::cout} w całym programie? Można użyć jednej z 2 metod:
\begin{itemize}
\item przekazać cały string - operator <{}< jest threadsafe - czyli jeśli przekażemy argument w całości zostanie wyświetlony w całości. Można wspomagać się lokalnym buforem <<przykładowy kod>>
\item użyć \code{ios\_base::register\_callback()} do synchronizacji i zablokowania muteksu. Działa także dla bibliotek zależnych.
\end{itemize}

\paragraph{Głębokie sekcje krytyczne}
Jak zablokować głębokie sekcje krytyczne? W sekcji krytycznej można zastosować asercję sprawdzającą czy nie jesteśmy już w innej sekcji krytycznej \code{Resource<>}. Można to łatwo zaimplementować z pomocą lokalnych zmiennych statycznych. <<przykładowy kod>>

\paragraph{Widoczność sekcji krytycznych}
Jedną z rzeczy które możnaby poprawić w klasie Resource jest widoczność. \inlineCpp{getGlobalCfg().critical\_section(...)} nie wyróżnia się w kodzie. Można poprawić to na 2 sposoby:
\begin{itemize}
\item makro\\
  \inlineCpp{CRITICAL\_SECTION(getGlobalCfg(),\{ ... \})}
\item klasa\\
  \inlineCpp{ResourceRef<CfgReader> res = getGlobalCfg();}\\
  \inlineCpp{res.critical\_section([](auto const\& x)\{...\})}
\end{itemize}

\theendnotes
\end{document}
