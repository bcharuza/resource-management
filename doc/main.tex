\documentclass[titlepage,twocolumn,a4paper,10pt,fleqn,leqno]{article}

\usepackage{hyperref}
\usepackage[utf8]{inputenc}
\usepackage[T1]{fontenc}
\usepackage{polski}
\usepackage{titlesec}
\usepackage{wrapfig}
\usepackage{listings}
\usepackage{blindtext}
\usepackage{xcolor}

\definecolor{codegreen}{rgb}{0,0.6,0}
\definecolor{codegray}{rgb}{0.5,0.5,0.5}
\definecolor{codepurple}{rgb}{0.58,0,0.82}
\definecolor{backcolour}{rgb}{0.95,0.95,0.95}

\lstdefinestyle{cpp-style}{
    backgroundcolor=\color{backcolour},
    commentstyle=\color{codegreen},
    keywordstyle=\color{magenta},
    numberstyle=\tiny\color{codegray},
    stringstyle=\color{codepurple},
    basicstyle=\ttfamily\scriptsize,
    breakatwhitespace=false,
    breaklines=true,
    captionpos=b,
    keepspaces=false,
    numbers=left,
    numbersep=1pt,
    showspaces=false,
    showstringspaces=false,
    showtabs=false,
    tabsize=2
}

\newcommand\includeCpp[2]{\noindent\lstinputlisting[language=C++,style=cpp-style,captionpos=t,title=\scriptsize#2]{#1/#2}}
\newcommand\includeText[2]{\noindent\lstinputlisting[captionpos=t,style=cpp-style,title=\scriptsize#2]{#1/#2}}

\newcommand\inlineBash[1]{[\nobreak\lstinline[language=bash,style=cpp-style,frame=lines]{$> #1}\nobreak]}

\newcommand\inlineCpp[1]{[\nobreak\lstinline[language=C++,style=cpp-style]{#1}\nobreak]}

\titleformat{\section}
  {\normalfont\large\bfseries}{\thesection.}{1em}{}
\titleformat{\subsection}
  {\normalfont\normalsize\itshape}{\thesubsection.}{1em}{}
\titleformat{\subsubsection}
            {\normalfont\normalsize\itshape}{\thesubsubsection.}{1em}{}

\usepackage{relsize,endnotes,xurl,hyperref}

\renewcommand\UrlFont{\rmfamily}

\newcommand{\note}[2]{#1\endnote{#2}}

\newcommand{\link}[1]{\url{#1}}

\newcommand{\mail}[1]{\href{mailto:#1}{#1}}

\newcommand{\Rsign}[1]{\protect\hspace{-.15em}\protect\raisebox{.4ex}{\smaller\smaller\smaller\textbf{#1}}}
\newcommand{\Cpp}[1]{\mbox{C\Rsign{+}\Rsign{+}\protect\hspace{-.15em}#1}}

\makeatletter
\renewcommand\paragraph{%
  \@startsection{paragraph}{4}{0mm}%
                {-\baselineskip}%
                {0.5\baselineskip plus 0.2\baselineskip minus 0.2\baselineskip}%
                {\normalfont\normalsize\bfseries}}
\makeatother

\newcommand\editNote[2]{#1\textit{\color{red}{<<#2>>}}}

\newcommand\code[1]{\texttt{#1}}
\newcommand\cname[1]{\code{#1}}
\newcommand\fname[1]{\code{#1()}}
\newcommand\tname[1]{\code{#1<>}}

\include{lib/title-page}

\setTitle{Współdzielenie zasobów}
\setTitleCatchphrases{\Cpp, Przetwarzanie współbieżne, Struktura programu}
\setTitlePhoto{images/pizza}
\setTitleDescription{Artykuł prezentuje wybrane metody dostępu do zasobów w przykładach wzorowanych na kodzie produkcyjnym. Autor omawia problematykę, a następnie wskazuje metody usystematyzowania dostępu do elementów współdzielonych.}

\setAuthor {Bartosz Charuza}
\authorPhoto {images/author}
\authorPosition {Regular Software Developer}
\authorProject {Alstom: Ebiscreen2000}
\authorWebpage {\link{https://github.com/bcharuza}}
\authorEmail {bcharuza@yahoo.com}
\authorDesc {Programista C++ specjalizujący się w wysokopoziomowej komunikacji między usługami sieciowymi. 5 letnie doświadczenie w sterownikach przemysłowych. Od roku pracuje nad systemami konroli kolei.}

\hypersetup{pdfauthor=\getAuthor,pdftitle=\getTitle,pdfpagemode=None,colorlinks=true,linkcolor=black,urlcolor=black}


\begin{document}
\begin{titlepage}
\maketitle
\end{titlepage}
\section{Wstęp}\label{sec:introduction}
\paragraph{Dlaczego napisałem ten artykuł?}
Nie lubię debugować i debuggerów w ogóle. Nie znam gorszego losu, niż debugowanie aplikacji wielowątkowych. Gdy siedzę przed debuggerem czuję się jak kot goniący laserową kropkę drwiącego ze mnie absolutu - z tym że kot dobrze się przy tym bawi.
Pewnego pięknego październikowego ranka dostałem zwrot ,,XYZ czasami nie działa'' to oznaczało, że czeka mnie kilka dni przedniej zabawy - moje autystyczne rand ez-vous z debuggerem i tysiącami linii kodu. Błędem okazało się użycie niewłaściwego muteksu - Straciłem 2 dni życia na prosty błąd. To był punkt przełomowy. Cisnąłem ,,Tak być nie może!'' i przepisałem tę bibliotekę. \note{Frustracja to potężna muza - przykładowo niedziałająca drukarka zmieniła losy świata.}{\link{https://en.wikipedia.org/wiki/Richard\_Stallman\#Events\_leading\_to\_GNU}}

\paragraph{Kto jest adresatem?}
Programiści \Cpp{}, od juniora który może znaleźć inspirację, po seniora który może chcieć wymagać podobnych rozwiązań w swoim projekcie. Zakładam że czytelnik posiada podstawową znajomość \Cpp{14} i wiedzę z zakresu przetwarzania wielowątkowego z użyciem mutexów biblioteki standardowej.

\paragraph{Jaki jest zakres artykułu?}
Publikacja prezentuje wzorzec \tname{Resource} i przypadki użycia. Omawiane będą kontrprzykłady, zagadnienia inicjacji i dostępu do zasobów. Nie opisuję narzędzi, ani podstaw wielowątkowości. Większość poruszanych tematów będzie związana z wykorzystaniem wspomnianego wzorcu.

Prezentowane metody sprawdzone są w aplikacjach klasy desktop przy użyciu mechanizmów biblioteki standardowej do kilkudziesięciu wątków. Nie będą poruszane rozwiązania dla aplikacji serwerowych, czy o wysokich wymaganiach wydajnościowych w których nie mam doświadczenia. Artykuł nie ma też zastosowania dla problematyki massive-parallelizm.

\paragraph{Struktura artykułu}
Próbowałem napisać ten artykuł w standardowej akademickiej strukturze problem-analiza-rozwiązanie. Skutkowało to zbyt rozciągłą formą wraz dojściem nowych wątków, dywagacji i przypadków szczególnych.

Zdecydowałem się zastosować formułę rozwiązanie-zastosowanie, gdzie przypadki użycia są podzielone na krótkie pary kontrprzykład-przykład. Mam nadzieję że pozwoli to zredukować rozmiar do minimum i dotrwać czytelnikowi do końca artykułu.

\section{Przygotowanie merytoryczne}\label{sec:reader-profile}
\paragraph{Zasób}
\note{Zasób}{\link{https://en.wikipedia.org/wiki/System\_resource}} to coś o ograniczonej dostępności. Zasoby w przeciwieństwie do informacji/wartości posiadają tożsamość. Zasobem może być urządzenie zewnętrzne, pamięć, czas procesora, a nawet obiekt.

\paragraph{Typ}
\note{Typ określa zestaw dozwolonych wartości i operacji obiektu.}{Bjarne Stroustrup: ,,Język \Cpp. Kompendium wiedzy.''}

\paragraph{Wartość}
\note{Wartość to zbiór bitów interpretowanych zgodnie z typem}{Bjarne Stroustrup: ,,Język \Cpp. Kompendium wiedzy.''}

\paragraph{Obiekt}
\note{Obiekt to miejsce w pamięci, w którym przechowywana jest wartość jakiegoś typu}{Bjarne Stroustrup: ,,Język \Cpp. Kompendium wiedzy.''}

\paragraph{Znajomość \Cpp{}14}
Większość kodu będzie napisana w \Cpp{}14. Czytelnik znający \Cpp{}11 nie powinien widzieć różnicy. Czytelnik znający \Cpp{}03 powinien być w stanie zrozumieć treść. Starsze standardy różnią się zbyt mocno aby artykuł mógł być łatwo przyswajalny.
\begin{itemize}
\item \link{https://en.cppreference.com/}
\item Bruce Eckel: ,,Thinking in \Cpp''
\item Bjarne Stroustrup: ,,Język \Cpp. Kompendium wiedzy.''
\end{itemize}

\paragraph{Przetwarzanie współbieżne}
Znajomość mutexów i podstaw wielowątkowości biblioteki standardowej \Cpp{} powinna być wystarczająca.
\begin{itemize}
\item Anthony Williams ,,Język \Cpp{} i przetwarzanie współbieżne w akcji''
\end{itemize}

\paragraph{Kompilacja i linkowanie}
By zrozumieć niektóre problemy czytelnik powinien umieć kompilować pliki źródłowe do plików binarnych a następnie potrafić je linkować w 2 osobnych krokach.

\paragraph{Programowanie szablonowe}
Część kodu jest kodem szablonowym. Do pełnego zorumienia implementacji należy poznać wykorzystanie \code{std::enable\_if<>} i pozostałych narzędzi programowania szablonowego udostępnianego przez bibliotekę standardową.

\section{Środowisko testowe}
\begin{tabular}{l r}
  Kompilator:   & GCC v9.3\\
  System:       & Ubuntu 20.04.2 LTS\\
  Architektura: & x86\_64
\end{tabular}

\section{Opis problemu}
Za punkt wyjścia przyjmiemy spreparowany kod złożony z 3 par plików \code{Logger.hpp/.cpp}, \code{MessageQueue.hpp/.cpp}, \code{CfgReader.hpp/.cpp} i pliku głównego \code{main.cpp}.
Przykłady są syntezą kodu produkcyjnego zastanego w kilku projektach z którymi pracowałem. 

\paragraph{Logger}
\includeCpp{../../examples/problem}{Logger.hpp}
\includeCpp{../../examples/problem}{Logger.cpp}
Powyższy przykład reprezentuje typowy kod C przełożony na grunt \Cpp{}. Często używane są bezpośrednio funkcje systemowe które nie dają gwarancji bezpieczeństwa lub gwarancja jest zależna od implementacji biblioteki jak \code{vfprintf}. W systemach zgodnych z POSIX funkcje printf są bezpieczne \link{https://pubs.opengroup.org/onlinepubs/9699919799/functions/V2\_chap02.html\#tag\_15\_09}, jednak dla Windows nie spotkałem się z taką gwarancją.

Dla porównania wg. \link{https://isocpp.org/files/papers/p0053r7.pdf} \code{std::cout} zapewnia bezpieczeństwo. Jest to jednak złudne gdyż z moich doświadczeń jest ono zapewnione tylko w obrębie pojedynczej operacji - \code{cout<<A<<B} może już zostać przerwane przez inny proces.

Dostęp do statycznego \code{s\_category} jest potencjalnie bezpieczny (nie licząc zbyt dużego indeksu). Przyczyną jest gwarantowane przez język automatyczne czyszczenie przechowywanych stringów po wyjściu z programu, i dostęp wyłącznie do odczytu - przez co obiekt posiada charakterystykę bardziej wartości niż obiektu. Problem dostępu do \code{s\_category} może wystąpić w tylko jednym przypadku - jeśli użytkownikowi uda się odwołać do obiektu jeszcze przed jego inicjacją - tzn. gdy \code{trace()} zostanie wywołany w kodzie inicjacyjnym w innej jednostce kompilacji, przed ustawieniem wartości \code{s\_category}.
\paragraph{CfgReader}
\includeCpp{../../examples/problem}{CfgReader.hpp}
\includeCpp{../../examples/problem}{CfgReader.cpp}
Reader opis

\paragraph{MessageQueue}
\includeCpp{../../examples/problem}{MessageQueue.hpp}
\includeCpp{../../examples/problem}{MessageQueue.cpp}
MessageQueue opis

\paragraph{main}
\includeCpp{../../examples/problem}{main.cpp}

\onecolumn
\theendnotes
\end{document}
