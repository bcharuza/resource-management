\section{Przykłady}\label{sec:examples}
Klasa \tname{Resource} pomogła mi w zwalczaniu błędów występujących w poprawianym kodzie:
\begin{itemize}
\item brak pokrycia zakresów mutexu i zasobu.
\item długie sekcje krytyczne.
\item jednoczesne zajmowanie wielu zasobów.
\item wyścigi danych
\item \note{wyścigi interface'u}{Anthony Williams ,,Język C++ i przetwarzanie współbieżne w akcji''}
\end{itemize}

\paragraph{CfgReader}
Pierwszym przykładem, który chcę omówić jest klasa \cname{CfgReader} -- silnie zredukowany przedstawiciel kodu który spotkałem w produkcji. Działanie jest banalne - po wczytaniu konfiguracji za pomocą metody \fname{LoadCfg} aktualizowane są pola statyczne klasy.
\includeCpp{../../examples}{CfgReader.hpp}
\includeCpp{../../examples}{CfgReader.cpp}
Można zauważyć następujące błędy:
\begin{itemize}
\item Pola mogą być odczytywane i zapisywane asynchronicznie doprowadzając do wyścigu danych.
\item \fname{LoadCfg} ma zakres instancji a modyfikowane pola są statyczne - to mylące i niebezpieczne.
\item Nie da się określić czy pola zostały zainicjowane czy jeszcze nie.
\item Przeładowanie konfiguracji nie jest operacją atomową.
\end{itemize}
Powyższy kod wymusza by konfiguracja była inicjowana jak najwcześniej podczas rozruchu aplikacji i by była niezmienna w dalszym ciągu działania programu - nic jednak nie chroni użytkownika przed wielokrotnym wywołaniem \fname{LoadCfg}. Korzystanie z pól statycznych klasy przed wywołaniem \fname{LoadCfg} musi być pilnowane przez programistę. Po inicjacji konfiguracja nie może być już przeładowana, i jest sprowadzona do zbioru wartości. Nie jesteśmy nawet w stanie kontrolować przypadkowej modyfikacji.

Jak użycie \tname{Resource} jest w stanie poprawić kod naszej klasy? 

\includeCpp{../../examples}{CfgReader-fix.hpp}
\includeCpp{../../examples}{CfgReader-fix.cpp}
Co zmieniłem?
Przeniosłem zarządzanie zasobem na wrapper \tname{Resource} i udostępniam przez funkcję globalną. Dostęp jest kontrolowany wewnętrznie. Pola statyczne zmieniłem na mapę - do przechowywania stanów współdzielonych jak konfiguracja mapy sprawdzają się najlepiej.

Funkcja \fname{LoadCfg} otrzymuje dostęp do mapy przez funkcję dostępową \fname{getCfg}, która udostępnia wspomnianą mapę owiniętą wewnątrz \tname{Resource}. Schemat aktualizacji konfiguracji jest standardowy - tworzymy lokalną strukturę, a następnie przypisujemy ją do zasobu w pojedynczym szybkim przypisaniu ograniczając czas wykonania sekcji krytycznej do minimum.

Dlaczego użyłem funkcji dostępowej zamiast umieścić zasób w zmiennej globalnej?
Ponieważ standard definiuje kolejność inicjacji zmiennej globalnej w zależności od pierwszeństwa jej wystąpienia, co oznacza że jest ona zależna od kolejności kompilacji/linkowania. Aby to objeść użyłem prostego rozwiązania - funkcji globalnej. Standard wymaga by pola statyczne były inicjowane przed jej wywołaniem tylko 1 raz - nawet w kodzie wielowątkowym. Kompilator automatycznie ustali kolejność inicjacji. Dodatkowo pozwala to na prostszą refaktoryzację i testowanie kodu.

\paragraph{Logger}
\cname{Logger} reprezentuje typowy kod C przełożony na grunt \Cpp{}. Jest to silnie zredukowany przykład wzorowany na popularnych leniwych logerach-samoróbkach, często przeplatanych niezależnymi operacjami wyjściowymi jak \fname{printf}, czy \code{std::cout} w innych miejscach programu. Jest to prawdziwy kłopot dla kodu wielowątkowego.
\includeCpp{../../examples}{Logger.hpp}
\includeCpp{../../examples}{Logger.cpp}
W powyższym kodzie są wykorzystywane funkcje systemowe które nie dają gwarancji bezpieczeństwa lub gwarancja jest zależna od implementacji biblioteki jak \fname{vfprintf}. \note{W systemach zgodnych z \mbox{POSIX} funkcje rodziny \fname{printf} są bezpieczne, jednak \fname{ctime} już nie musi}{\link{https://pubs.opengroup.org/onlinepubs/9699919799/functions/V2\_chap02.html}}. Dla Windows nie spotkałem się z taką gwarancją. Używając biblioteki C trzeba szczególnie zwracać uwagę czy wybrany element jest przystosowany do użycia w środowisku wielowątkowym. Dostęp do konfiguracji jest niechroniony i może powodować liczne błędy wyścigu do danych.

Jak ta kwestia wygląda w \Cpp{} ?
Dla porównania wg. \note{propozycji p0053r7}{\link{https://isocpp.org/files/papers/p0053r7.pdf}} standard zapewnia bezpieczeństwo dla strumieni wyjściowych takich jak \vname{std::cout}. Bezpieczeństwo jest gwarantowane tylko dla pojedynczej operacji, więc \code{cout<<A<<B} może zostać przerwane przez inny proces między wywołaniem \code{<\,<A} i \code{<\,<B}. Należy też uważać na starsze wersje biblioteki \lib{stdc++}, która mogła niepoprawnie obsługiwać wielowątkowe operacje na obiektach strumieni. Standard \Cpp{20} udostępnia już wrapper \cname{basic\_osyncstream}.

Spróbujmy przeprojektować powyższy kod, tak by zachować wygodę \fname{vfprintf} i zagwarantować bezpieczeństwo wielowątkowe.

\includeCpp{../../examples}{Logger-fix.hpp}
\includeCpp{../../examples}{Logger-fix.cpp}
Co zmieniono?
Funkcję \fname{\tname{trace}} kompletnie przepisałem, tak by używała tylko biblioteki \Cpp{}. Wyświetlenie wiadomości odbywa się w pojedynczym wywołaniu \fname{ostream::operator <\,<} by mieć pewność że wydruk jest bezpieczny ze względu na wątki. Zapełnianie buforu \cname{oss} odbywa się w pamięci lokalnej, więc wszelkie interakcje z zasobem \vname{std::cout} są ograniczone do minimum.

Funkcja \fname{trace\_sysinfo} jest odpowiedzialna za zapisanie nagłówku wiadomości i przeniesiono ją do \file{Logger-fix.cpp} by ograniczyć zależności \file{Logger-fix.hpp}.

\fname{\tname{trace\_impl}} odpowiada za rekurencyjne wywołania szablonowe o zmiennej liczbie argumentów. Jest to rozwiązanie dające możliwości variadic functions znanych z C i jednocześnie zachowujące ścisłą kontrolę typów znaną z \Cpp{}. Dodatkowo użytkownik może definiować własne przeciążenia \fname{\tname{trace\_print}} by dodać własną obsługę typów złożonych. Funkcje pomocnicze umieszczono w tzw. anonimowej przestrzeni nazw, by uniemożliwić dostęp do nich spoza pliku.

Warto zauważyć że dostęp do konfiguracji odbywa się przez \fname{getCfg} atomowo w krótkiej sekcji krytycznej. Obsługa sekcji krytycznej jest niemal niezauważalna dla użytkownika kodu i objawia się użyciem lambdy.

\paragraph{MessageHandler}
Ostatnim przykładem jest klasa \cname{MessageHandler}, która też jest inspirowana kodem produkcyjnym i przedstawia częste bolączki niewłaściwie zaprojektowanego układu klas.
\includeCpp{../../examples}{MessageHandler.hpp}
\includeCpp{../../examples}{MessageHandler.cpp}
W powyższym kodzie klasa \cname{MsgHandler} pełni funkcję centralnej koordynacji przepływu wiadomości. Zawiera jeden globalny interface we/wy \cname{MQClient} z którym związuje się przez wymianę adresów. Już na tym etapie widać pierwszą pułapkę - adres nic nie mówi o czasie życia obiektu i gdy jeden z obiektów przestanie istnieć narażamy się na naruszenie ochrony pamięci. Drugim trudniej zauważalnym problemem jest pełzająca spagettyfikacja kodu - tworzenie takich odniesień 1-1 powoduje że adres zasobu będzie przekazywany między instancjami klienckimi. Modyfikacja i zrozumienie zależności takiego kodu będą z czasem coraz trudniejsze. Aby tego uniknąć zasoby powinny być umieszczane w globalnych repozytoriach.

Co gorsze połączenia wcale nie muszą być symetryczne. Po wielokrotnym wywołaniu \fname{setClient} dla różnych instancji \cname{MsgHandler} otrzymamy wiele instancji \cname{MQClient} wskazujących na różne \cname{MsgHandler}, które z kolei wskazują na ostatnią powiązaną instancję \cname{MQClient}. To nie może skończyć się dobrze.

Poza we/wy klasa \cname{MsgHandler} zarządza także mechanizmem subskrypcji które polegają na wywołaniu callbacku po otrzymaniu nowej wiadomości. Subskrypcje są współdzielone między instancjami \cname{MsgHandler}, a subskrybenci są informowani kolejno wewnątrz metody \fname{NotifyAll} - to powoduje że zablokowanie jednego opóźnienia generowane przez każdego z odbiorców się kumulują. Pomijając możliwe zakleszczenia powodowane wywołaniem kodu klienckiego pewne są też długie czasy wykonania w wątku wywołującym. Przeniesienie obsługi notyfikacji do osobnego wątku mogłoby poprawić responsywność, jednak jest to obecnie spychane na użytkownika klasy.

Ostatnim z obowiązków klasy \cname{MsgHandler} jest zarządzanie kolejką wiadomości która także dzielona między instancjami. Sama klasa \cname{MsgQueue} jest zrealizowana prawidłowo, jednak nie jest ona przystosowana do kodu wielowątkowego. Co prawda \cname{MsgQueue} jest chroniona przez \cname{MsgHandler}, jednak ten zezwala na tzw. wyścig interface'u. Wyścig interface'u polega na tym że klasa zakłada niejawną transakcję - tj. że klient musi pamiętać jaki jest stan zasobu między 2 kolejnymi operacjami. Zapamiętany stan może być w międzyczasie zmodyfikowany przez inny wątek. Przykładowo mając 2 wątki \code{A} i \code{B} i współdzielony obiekt \code{MsgHandler x}:

\begin{tabular}{l | l}
  \code{A} & \code{B}\\
  \hline
  \code{if(x.isEmpty())} & \code{if(x.isEmpty())}\\
  \code{  x.sendTop()} & \code{  x.sendTop()}
\end{tabular}

\noindent
Dla kolejności AABB i BBAA powyższy kod może być wykonany poprawnie, jednak dla ABAB, BABA, ABBA i BAAB może doprowadzić do załamania aplikacji (gdy w kolejce jest 1 element), pomimo że zarówno \fname{isEmpty} jak i \fname{sendTop} są wykonywane jako sekcje krytyczne.

Kończąc, pozostaje zauważyć że jak na ironię jedynym elementem który nie jest dzielony między instancjami jest mutex, który powinien chronić elementy współdzielone przed jednoczesnym dostępem. Obecnie nie spełnia on swojej roli.

Powyższy kod skompiluje się, jednak klasa nie będzie działać. Błąd jest prosty, więc proponuję by czytelnik spróbował go odszukać, aby zrozumieć dlaczego zdecydowałem się napisać \tname{Resource}.

Strukturę tego programu można szybko poprawić za pomocą globalnych repozytoriów zasobów i wrapperu \tname{Resource}. Pytaniem otwartym jest - co potraktujemy jako zasób? Aby to zrobić trzeba przeanalizować przypadki użycia i przepływ informacji. Wiemy że:
\begin{itemize}
\item jedyną klasą posiadającą dane współdzielone jest \cname{MsgHandler}.
\item Każda z instancji \cname{MQClient} może komunikować się z jedną instancją \cname{MsgHandler}
\item Każda z instancji \cname{MsgHandler} musi komunikować się z ostatnią skonfigurowaną \cname{MQClient}.
\end{itemize}

Wyłania się z tego obraz systemu muksującego (nie mylić z mutexem) z buforowaniem. System muksujący kieruje wiele sygnałów wejściowych na jedno wyjście. Wobec tego powinniśmy posiadać możliwość tworzenia nieograniczonej ilości instancji \cname{MQClient} i posiadać możliwość zarejestrowania 1 instancji jako wyjścia \cname{MsgHandler}. Moim zdaniem taki hub nie powinien móc współdzielić pamięci między swoimi instancjami więc subskrypcje i kolejka powinny być przechowywane lokalnie. Metody \cname{MsgHandler} będą wykonywane współbieżnie, więc instancje tej klasy traktować należy jako zasób. Z kolei \cname{MQClient} chcemy używać lokalnie dla wątku. Co jednak się stanie gdy zarejestrujemy obiekt w dwóch \cname{MsgHandler}? Jak poinformować \cname{MsgHandler} że obiekt przestał istnieć? Dostęp do \cname{MQClient} także należałoby przeprojektować.
\includeCpp{../../examples}{MessageHandler-fix.hpp}
\includeCpp{../../examples}{MessageHandler-fix.cpp}
Co się zmieniło? Najważniejszą zmianą jest dodanie \fname{initializeMsgHandler} i \fname{accessMsgHandler} które przyjmują formę prostego repozytorium dla obiektów \cname{MsgHandler}.

W \cname{MQClient} zmieniłem sposób przechowywania muxu z prostego wskaźniku na \tname[Resource]{shared\_ptr}. Wrapperem \tname{Resource} chronimy osobno wejście i wyjście narażone na współbieżny dostęp z \cname{MsgHandler} - zauważ jak naturalne staje się odseparowanie sekcji krytycznych w \fname{MQClient::receive}. Pobranie instancji \cname{MsgHandler} odbywa się tylko z repozytorium zasobów. Warto nadmienić, że wykorzystano tu bezpieczeństwo zliczania odniesień \tname{shared\_ptr} dla wywołań współbieżnych. Dostępem administruje zaś wrapper \tname{Resource}. Alternatywnym rozwiązaniem może być też przechowywanie nazwy/id instancji muxu i każdorazowe pobieranie instancji z repozytorium.

\cname{MsgHandler} wchłonął klasę \cname{Subscription} która wyłącznie obciążała użytkownika - teraz operacje nadawania unikalnego ID odbywają się we wnętrzu klasy w \fname{getNewId}. Wszystkie pola klasy są polami instancji dzięki czemu unikamy konieczności pilnowania dostępu do nich.

Z \cname{MsgHandler} usunięto mutex instancji. Nie jest to rozwiązanie czysto pozytywne - ma wady i zalety. Wadą jest porzucenie kontroli nad dostępem do klasy - przez co dla środowiska wielowątkowego musi być zawsze otoczona przez wrapper \tname{Resource}. Niewątpliwą zaletą jest jednak brak konieczności zajmowania się dostępem wielowątkowym. Usunięto mutex, którego nie trzeba pilnować w kodzie i prawdopodobieństwo wystąpienia wyścigu interface'u zostało znacznie zredukowane.

Oczywiście znacznie lepszym rozwiązaniem byłoby przebudowanie prezentowanego wcześniej \cname{MsgHandler} tak aby klasa realizowała interface atomowy kontrolowany przez własne mutexy czy operacje bezkolizyjne. Klasy takie muszą zostać jednak przygotowane przez doświadczonego programistę specjalizującego się w kodzie współbieżnym, który zastosowałby zresztą inne rozwiązania architektoniczne.
