\section{Przykłady}\label{sec:examples}
Klasa \tname{Resource} wspomnianą w sekcji \ref{sec:resource} napisałem by zwalczać błędy w istniejącym kodzie:
\begin{itemize}
\item brak pokrycia zakresów muteksu i zasobu.
\item długie sekcje krytyczne.
\item jednoczesne zajmowanie wielu zasobów -- w tym obsługa operacji I/O.
\item wyścigi interface'u.
\item rozległe sekcje krytyczne.
\end{itemize}

\editNote{}{TODO: dalej od tego momentu}
%% \paragraph{CfgReader}
%% \code{CfgReader} zmodyfikowano by zamiast pól statycznych posiadał pola lokalne dla instancji. Po usunięciu pól statycznych klasa będzie teraz prostsza w testowaniu. Dostęp do globalnej instancji odbywa się przez \code{getGlobalCfg()}. Poprawna inicjacja jest gwarantowana przez standard.
%% \includeCpp{../../examples/resource}{CfgReader.hpp}
%% \includeCpp{../../examples/resource}{CfgReader.cpp}

%% \paragraph{MsgHandler}
%% Nowy wrapper ma też zastosowanie w klasie \code{MsgHandler}. Zastosowanie szablonu \code{Resource<>} uporządkuje też błędne zakresy i wewnętrzne zakleszczenia.
%% \includeCpp{../../examples/resource}{MessageHandler.hpp}
%% \includeCpp{../../examples/resource}{MessageHandler.cpp}
%% Należy zauważyć że \code{MQClient} nie używa już muteksu - jest on już niepotrzebny gdyż to czy instancja ma służyć do użytku wielowątkowego jest definiowane przez umieszczenie jej wewnątrz \code{Resource<>}. Następną ważną różnicą jest zmiana metody \code{MsgHandler::setClient()} - instancja zasobu nie jest już upubliczniana poza \code{MsgHandler} - klasa jest właścicielem wszystkich zawieranych przez siebie zasobów.

%% Kod metod klasy \code{MsgHandler} wydaje się czystszy i udało się usunąć zakleszczenia. Zasada jest prosta - pobieramy zasób - wykonujemy minimalny zestaw instrukcji w sekcji krytycznej i wracamy.
%% Sekcje krytyczne bazują na mechanizmie funkcji lambda wprowadzonych w \Cpp{11} - działają one jak obiekty funkcyjne i mają prosty format: \code{[<zmienne zewnętrzne>](auto\&~x)\{<zestaw~instrukcji>\}}. Wynik return można przekazać poza sekcję krytyczną poprzez przypisanie wyniku do zmiennej.

%% Wszystkie funkcje \code{MsgHandler} są teraz statyczne i tworzenie instancji jest niepotrzebne. Okazuje się że Sama klasa \code{MsgHandler} jest niepotrzebna i można ją rozbić na niezależne funkcje globalne. Alternatywnie można zezwolić by każda instancja posiadała lokalne pola \code{MQClient}, \code{MsgQueue}, czy listę subskrybcji - wtedy należy ściągnąć wrappery \code{Resource<>} z lokalnych pól instancji i nałożyć go na współdzielone instancje \code{MsgHandler}.

%% Funkcja \code{accessClient()} pokazuje jak w prosty sposób zrealizować opóźnioną inicjację - wystarczy stworzyć zasób dla \code{unique\_ptr}.

%% \paragraph{main}
%% Main został lekko zmieniony - warto zwrócić uwagę jak inicjowany jest teraz \code{CfgReader}. Zamiast ustawiania globalnych pól - inicjowany jest lokalny obiekt, a następnie jest kopiowany do przestrzeni współdzielonej. W ten sposób zajmujemy tylko jeden zasób jednocześnie - pierw plik, a później obiekt współdzielony. 
%% \includeCpp{../../examples/resource}{main.cpp}
%% Aplikacja jest teraz stabilna - nie ma wyścigów danych. Nie ma błędów kolejności inicjacji zmiennych statycznych. Klasa \code{MsgHandler} nadal pozwala na wyścigi związane z nieatomowym interfacem. \code{trace()} nadal wykorzystuje mechanizmy C, co powoduje błędną interpretację danych podczas wyświatlenia konfiguracji.

%% Kod tak jak poprzednio pos kompilowaniu mozna uruchomić za pomocą \inlineBash{ctest resource-test1} (lub \inlineBash{<resource-binary> test1.cfg} gdy kompilujemy bez CMake).

%% \paragraph{\code{Resource<unique\_ptr<>{}>} daje wsparcie dla opóźnionej inicjacji.}
