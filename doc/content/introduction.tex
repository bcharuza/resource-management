\section{Wstęp}\label{sec:introduction}
\paragraph{Dlaczego napisałem ten artykuł?}
Na przestrzeni lat napotkałem wiele aplikacji które były trudne w czytaniu i utrzymaniu. Ten tekst jest jednym z serii artykułów zawierających propozycje rozwiązań które dotychczas stosowałem.
W dalszej części skupię się na abstrakcji zasobu, którego użycie usystematyzowało dostęp do zasobów i uprościło strukturę moich programów.

\paragraph{Kto jest targetem?}
Programiści \Cpp{} aplikacji wielowątkowych. Zakładam że czytelnik posiada podstawową znajomość \Cpp{14} i wiedzę z zakresu przetwarzania wielowątkowego z użyciem muteksów biblioteki standardowej.

\paragraph{Jaki jest zakres artykułu?}
Publikacja prezentuje wzorzec \tname{Resource} i przypadki użycia. Omawiane będą też kontrprzykłady, zagadnienia inicjacji i dostępu do zasobów.  Nie opisuję narzędzi, ani podstaw wielowątkowości.

Prezentowane metody sprawdzone są w aplikacjach klasy desktop przy użyciu mechanizmów biblioteki standardowej do kilkudziesięciu watków. Nie będą poruszane rozwiązania dla aplikacji serwerowych, czy o wysokich wymaganiach wydajnościowych w których nie mam doświadczenia. Artykuł nie ma też zastosowania dla problematyki massive-parallelizm.

Artykuł skupia się na czytelności kodu i technik zachęcających użytkowników kodu do zachowania czytelności. Bezpieczeństwo będzie poruszane tylko w ramach danych przykładów i nie omawiane kompleksowo. Wydajność rozwiązań nie będzie poruszana.

\paragraph{Struktura artykułu}
Próbowałem napisać ten artykuł w standardowej akademickiej strukturze problem-analiza-rozwiązanie. Skutkowało to zbyt rozciągłą formą wraz dojściem nowych wątków, dywagacji i przypadków szczególnych.

Zdecydowałem się zastosować formułę rozwiązanie-zastosowanie, gdzie zastosowanie jest podzielone na krótkie pary kontrprzykład-przykład. Mam nadzieję że pozwoli to zredukować rozmiar do minimum i pozwoli dotrwać czytelnikowi do końca artykułu.
