\section{Wstęp}\label{sec:introduction}
\paragraph{Dlaczego napisałem ten artykuł?}
Historia tego artykułu miała początek pewnego pięknego październikowego ranka, krótko po załączeniu komputera otrzymałem radosne przywitanie. ,,Hello. How are you?''.\\
Poniżej czaiły się 3 kropki grozy\\
...\\
\\
,,Sometimes X doesn't work'' - oto mój bilet na autystyczne randez-vous z tysiącami linii kodu.
,,Sometimes'' jest szczególnie złowieszczym słowem, gdyż ujawnia wyścig - teraz będę  godzinami gonił czerwoną kropkę drwiącego losu. Po 2 dniach galerniczej przygody odkryłem skarb! Wyścig był powodowany zajęciem niewłaściwego muteksu. Nastąpił przełom - \note{niczym oporna drukarka, która zmieniła losy świata}{\link{https://en.wikipedia.org/wiki/Richard\_Stallman\#Events\_leading\_to\_GNU}}, ten niedziałający muteks zmobilizował mnie do przepisania biblioteki i napisania serii artykułów.

\paragraph{Kto jest adresatem?}
Programiści \Cpp{}, od juniora który może znaleźć inspirację, po seniora który może chcieć wymagać podobnych rozwiązań w swoim projekcie. Zakładam że czytelnik posiada podstawową znajomość \Cpp{14} i wiedzę z zakresu przetwarzania wielowątkowego z użyciem mutexów biblioteki standardowej.

\paragraph{Jaki jest zakres artykułu?}
Publikacja prezentuje wzorzec \tname{Resource} i przypadki użycia. Omawiane będą kontrprzykłady, zagadnienia inicjacji i dostępu do zasobów. Nie opisuję narzędzi, ani podstaw wielowątkowości. Większość poruszanych tematów będzie związana z wykorzystaniem wspomnianego wzorcu.

Prezentowane metody sprawdzone są w aplikacjach klasy desktop przy użyciu mechanizmów biblioteki standardowej do kilkudziesięciu wątków. Nie będą poruszane rozwiązania dla aplikacji serwerowych, czy o wysokich wymaganiach wydajnościowych w których nie mam doświadczenia. Artykuł nie ma też zastosowania dla problematyki massive-parallelizm.

\paragraph{Struktura artykułu}
Próbowałem napisać ten artykuł w standardowej akademickiej strukturze problem-analiza-rozwiązanie. Skutkowało to zbyt rozciągłą formą wraz dojściem nowych wątków, dywagacji i przypadków szczególnych.

Zdecydowałem się zastosować formułę rozwiązanie-zastosowanie, gdzie przypadki użycia są podzielone na krótkie pary kontrprzykład-przykład. Mam nadzieję że pozwoli to zredukować rozmiar do minimum i dotrwać czytelnikowi do końca artykułu.
