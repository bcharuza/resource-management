\section{Wstęp}\label{sec:introduction}
\paragraph{Dlaczego napisałem ten artykuł?}
Historia tego artykułu miała początek pewnego pięknego październikowego ranka, krótko po załączeniu komputera otrzymałem w komunikatorze radosne przywitanie: ,,Hello. How are you?''.\\
...\\
Tajemnicze 3 kropki przybrały ostateczną formę.\\
,,Sometimes X doesn't work'' - Gratulacje! Zdobyłeś główną nagrodę - autystyczne randez-vous z tysiącami linii kodu! Podekscytowany nadchodzącą przygodą zaparzyłem herbatę. W ramach dykresji dodam że wolę spędzić 2 godziny na pisaniu testów niż 1 na debugowaniu. Lubię gdy coś dzieje się ,,Always'' lub ,,Never'' - to takie pięne matematyczne sformułowania, natomiast ,,Sometimes'' budzi we mnie grozę - ,,Sometimes'' oznacza zachowania niezdefiniowane.

Po 2 dniach galerniczej przygody odkryłem skarb - wyścig powodowany zajęciem niewłaściwego muteksu. Ten niedziałający muteks był \note{niczym oporna drukarka, która zmieniła losy świata}{\link{https://en.wikipedia.org/wiki/Richard\_Stallman\#Events\_leading\_to\_GNU}}. ,,Tak dalej być nie może'' postanowiłem i przepisałem tę bibliotekę. Jest to pierwsza relacja z mojej walki.

\paragraph{Kto jest adresatem?}
Programiści \Cpp{} - od juniora który może poznać techniki korzystania z muteksów, po seniora który może zechcieć zastosować podobne rozwiązania w swoim projekcie. Zakładam że czytelnik posiada podstawową znajomość \Cpp{14} i wiedzę z zakresu przetwarzania wielowątkowego z użyciem mutexów biblioteki standardowej.

\paragraph{Jaki jest zakres artykułu?}
Publikacja prezentuje wzorzec \tname{Resource} i przypadki użycia. Omawiane będą kontrprzykłady, zagadnienia inicjacji i dostępu do zasobów. Nie opisuję narzędzi, ani podstaw wielowątkowości. Większość poruszanych tematów będzie związana z wykorzystaniem wspomnianego wzorcu.

Prezentowane metody sprawdzone są w aplikacjach klasy desktop przy użyciu mechanizmów biblioteki standardowej do kilkudziesięciu wątków. Nie będą poruszane rozwiązania dla aplikacji serwerowych, czy o wysokich wymaganiach wydajnościowych w których nie mam doświadczenia. Artykuł nie ma też zastosowania dla problematyki massive-parallelizm.

\paragraph{Struktura artykułu}
Próbowałem napisać ten artykuł w standardowej akademickiej strukturze problem-analiza-rozwiązanie. Skutkowało to zbyt rozciągłą formą wraz dojściem nowych wątków, dywagacji i przypadków szczególnych.

Zdecydowałem się zastosować formułę rozwiązanie-zastosowanie, gdzie przypadki użycia są podzielone na krótkie pary kontrprzykład-przykład. Mam nadzieję że pozwoli to zredukować rozmiar do minimum i dotrwać czytelnikowi do końca artykułu.
