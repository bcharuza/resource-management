\section{Wstęp}\label{sec:introduction}
\paragraph{Dlaczego napisałem ten artykuł?}
Nie lubię debugować i debuggerów w ogóle. Nie znam gorszego losu, niż debugowanie aplikacji wielowątkowych. Gdy siedzę przed debuggerem czuję się jak kot goniący laserową kropkę drwiącego ze mnie absolutu - z tym że kot dobrze się przy tym bawi.
Pewnego pięknego październikowego ranka dostałem zwrot ,,XYZ czasami nie działa'' to oznaczało, że czeka mnie kilka dni przedniej zabawy - moje autystyczne rand ez-vous z debuggerem i tysiącami linii kodu. Błędem okazało się użycie niewłaściwego muteksu - Straciłem 2 dni życia na prosty błąd. To był punkt przełomowy. Cisnąłem ,,Tak być nie może!'' i przepisałem tę bibliotekę. \note{Frustracja to potężna muza - przykładowo niedziałająca drukarka zmieniła losy świata.}{\link{https://en.wikipedia.org/wiki/Richard\_Stallman\#Events\_leading\_to\_GNU}}

\paragraph{Kto jest adresatem?}
Programiści \Cpp{}, od juniora który może znaleźć inspirację, po seniora który może chcieć wymagać podobnych rozwiązań w swoim projekcie. Zakładam że czytelnik posiada podstawową znajomość \Cpp{14} i wiedzę z zakresu przetwarzania wielowątkowego z użyciem mutexów biblioteki standardowej.

\paragraph{Jaki jest zakres artykułu?}
Publikacja prezentuje wzorzec \tname{Resource} i przypadki użycia. Omawiane będą kontrprzykłady, zagadnienia inicjacji i dostępu do zasobów. Nie opisuję narzędzi, ani podstaw wielowątkowości. Większość poruszanych tematów będzie związana z wykorzystaniem wspomnianego wzorcu.

Prezentowane metody sprawdzone są w aplikacjach klasy desktop przy użyciu mechanizmów biblioteki standardowej do kilkudziesięciu wątków. Nie będą poruszane rozwiązania dla aplikacji serwerowych, czy o wysokich wymaganiach wydajnościowych w których nie mam doświadczenia. Artykuł nie ma też zastosowania dla problematyki massive-parallelizm.

\paragraph{Struktura artykułu}
Próbowałem napisać ten artykuł w standardowej akademickiej strukturze problem-analiza-rozwiązanie. Skutkowało to zbyt rozciągłą formą wraz dojściem nowych wątków, dywagacji i przypadków szczególnych.

Zdecydowałem się zastosować formułę rozwiązanie-zastosowanie, gdzie przypadki użycia są podzielone na krótkie pary kontrprzykład-przykład. Mam nadzieję że pozwoli to zredukować rozmiar do minimum i dotrwać czytelnikowi do końca artykułu.
