\section{Przygotowanie merytoryczne}
\paragraph{Zasób}
coś o ograniczonej dostępności. Zasoby w przeciwieństwie do informacji/wartości posiadają tożsamość. Zasobem może być urządzenie zewnętrzne, miejsce w pamięci, czas procesora, a nawet obiekt.

\paragraph{Typ}
Czym jest typ? - Stroustrup

\paragraph{Wartość}
Czym jest wartość? - Strostrup

\paragraph{Obiekt}
Czym jest obiekt? - Strostrup

\paragraph{Znajomość \Cpp{}14}
Większość kodu będzie używało standardu \Cpp{}14. Czytelnik znający \Cpp{}11 nie powinien widzieć różnicy. Czytelnik znający \Cpp{}03 powinien być w stanie zrozumieć treść. Starsze standardy różnią się zbyt mocno aby artykuł mógł być łatwo przyswajalny.
\begin{itemize}
\item \link{https://en.cppreference.com/}
\item Bruce Eckel: ,,Thinking in \Cpp''
\item Bjarne Stroustrup: ,,Język \Cpp. Kompendium wiedzy.''
\end{itemize}

\paragraph{Przetwarzanie współbieżne}
Znajomość mutexów i podstaw wielowątkowości biblioteki standardowej \Cpp{} powinna być wystarczająca.
\begin{itemize}
\item Anthony Williams ,,Język \Cpp{} i przetwarzanie współbieżne w akcji''
\end{itemize}

\paragraph{Kompilacja i linkowanie}
By zrozumieć niektóre problemy czytelnik powinien umieć kompilować pliki źródłowe do plików binarnych a następnie potrafić je linkować w 2 osobnych krokach.

\paragraph{Programowanie szablonowe}
Część kodu jest kodem szablonowym. Do pełnego zorumienia implementacji należy poznać wykorzystanie \code{std::enable\_if<>} i pozostałych narzędzi programowania szablonowego udostępnianego przez bibliotekę standardową.
