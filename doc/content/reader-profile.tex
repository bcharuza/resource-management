\section{Ustalenia}\label{sec:reader-profile}
\paragraph{Oznaczenia}
\begin{tabular}{l|l}
  \lib{Nazwy bibliotek} & \file{Nazwy plików} \\
  \hline
  \fname{Nazwy\_funkcji} & \tname{Nazwy\_szablonów} \\
  \hline
  \vname{Nazwy\_zmiennych} & \cname{Nazwy\_klas}
\end{tabular}
\paragraph{Zasób}
\note{Zasób to coś o ograniczonej dostępności.}{\link{https://en.wikipedia.org/wiki/System\_resource}} Zasoby w przeciwieństwie do informacji/wartości posiadają tożsamość. Zasobem może być urządzenie zewnętrzne, pamięć, czas procesora, a nawet obiekt.

\paragraph{Typ}
\note{Typ określa zestaw dozwolonych wartości i operacji obiektu.}{Bjarne Stroustrup: ,,Język \Cpp. Kompendium wiedzy.''}

\paragraph{Wartość}
\note{Wartość to zbiór bitów interpretowanych zgodnie z typem}{Bjarne Stroustrup: ,,Język \Cpp. Kompendium wiedzy.''}

\paragraph{Obiekt}
\note{Obiekt to miejsce w pamięci, w którym przechowywana jest wartość jakiegoś typu}{Bjarne Stroustrup: ,,Język \Cpp. Kompendium wiedzy.''}

\paragraph{Sekcja krytyczna}
\note{Sekcją krytyczną jest zestaw instrukcji chroniony przed jednoczesnym wywołaniem względem danego zasobu}{\link{https://en.wikipedia.org/wiki/Critical\_section}}

\paragraph{Znajomość \Cpp{14}}
Używan standardu \Cpp{14}. Czytelnik znający \Cpp{11} nie powinien widzieć różnicy. Czytelnik znający \Cpp{03} powinien być w stanie zrozumieć treść. Starsze standardy różnią się zbyt mocno aby artykuł mógł być łatwo przyswajalny.

\paragraph{Przetwarzanie współbieżne}
Znajomość mutexów i podstaw wielowątkowości biblioteki standardowej \Cpp{} powinna być wystarczająca.

\paragraph{Programowanie szablonowe}
Część kodu jest kodem szablonowym. Do zrumienia mechanizmów należy znać przynajmniej podstawy programowania szablonowego i narzędzia szablonowe biblioteki standardowej. Pełne zrozumienie implementacji wymaga znajomości \note{szablonów o zmiennej liczbie argumentów}{variadic templates}, i tzw. \note{uniwersalnych referencji}{\link{https://isocpp.org/blog/2012/11/universal-references-in-c11-scott-meyers}}.
