\section{Dodatki}\label{sec:extras}

\paragraph{Rekurencja}
Wywołania rekurencyjne dla standardowych zasobów będą się samoistnie zakleszczać. Aby zezwolić na wielokrotne zajmowanie tego samego muteksu w jednym wątku, należy użyć \cname{std::recursive\_mutex}
\begin{lstlisting}[language=C++,style=cpp-style,aboveskip=2mm]
  template <typename T> using RResource
  = Resource<T,std::recursive_mutex>
\end{lstlisting}

\paragraph{Równoczesne zajęcie 2 zasobów}
Jednoczesne zajmowanie więcej niż jednego zasobu powoduje niepotrzebne obciążenia i może prowadzić do trudnych w wykryciu zakleszczeń. Jak zdiagnozować podwójne jednoczesne zajęcia zasobów? Można do klasy \tname{Resource} dostarczyć mutex owinięty w licznik ilości zastosowanych blokad w lokalnym wątku.
\includeCpp{../../examples}{debug-mutex.hpp}
Następnie taki muteks można wykorzystać przy definiowaniu
\begin{lstlisting}[language=C++,style=cpp-style,aboveskip=2mm]
  template <typename T> using DebugResource
  = Resource<T,DebugMutex<std::mutex>>
\end{lstlisting}

\paragraph{Dostęp asymetryczny}
Dostęp asymetryczny można zrealizować bardzo podobnie przy użyciu \cname{std::shared\_mutex} i lekko modyfikując \tname{Resource}
\includeCpp{../../examples}{AsymResource.hpp}
