\section{Abstrakcja zasobu}\label{sec:resource}
Dlaczego zasoby sprawiają tyle problemów? Ponieważ są wyjątkowe i w przeciwieństwie do informacji - niepomnażalne. Dla przykładu, ten artykuł można rozesłać do milionów urządzeń bez żadnych strat dla oryginalnej treści - będzie ona nieodróżnialna, jednak moje odręczne notatki są wyjątkowe, niepowtarzalne, a przy próbie przesłania ich do każdego z czytelników prawdopodobnie nigdy nie dotarłyby do ostatniego w kolejce.

Dokładnie - w kolejce - przy próbie jednoczesnego dostępu moje notatki skończyłaby nie lepiej niż zdobycz w corocznej bitwie o karpia. Oczywiście my - programiści - z racji chronicznych niedostatków tkanki mięśniowej jesteśmy zmuszeni do zaniechania brutalnej rywalizacji na rzecz usystematyzowanych reguł dostępu. Wg. etykiety czekamy aż zasób będzie dostępny - rezerwujemy go - używamy - i udostępniamy pozostałym gdy skończymy. W skrócie stosujemy dostęp wykluczający - mutexy.

Łatwo powiedzieć trudniej zrobić. Pomimo że wybiera je 9/10 programistów, mutexy posiadają zestaw wad:
\begin{itemize}
\item są duże -- std::mutex osiąga 40 bajtów
\item są na niskim poziomie abstrakcji -- musimy o nich cały czas pamiętać, zawsze i wszędzie.
\item powodują wzajemne wykluczenie wątków -- łatwo stają się wąskim gardłem aplikacji.
\item uwielbiają się zakleszczać na wszelkie możliwe sposoby.
\end{itemize}
Pomimo powyższych wad posiadają 2 ogromne zalety - są proste w obsłudze i wszechstronne. Zrozumienie mechanizmu muteksu zajmuje kilka minut, po których świeżo upieczony absolwent może zostać posłany na front walki o informatyzację kraju.

\paragraph{Resource}
\tname{Resource} jest krótkim szablonem który obmyśliłem jako metodę systematyzacji zarządzania zasobami przy pomocy mutexów. Szablon powstał na mój własny użytek podczas refaktoryzacji kodu jednej z bibliotek komunikacji sieciowej. Ma on na celu silnie związać zasób z osłaniającym go mutexem, zautomatyzować zajmowanie i zwalnianie zasobu, i stanowić ramę popychającą użytkownika w kierunku poprawnych metod dostępu.

\includeCpp{../../examples}{Resource.hpp}

Klasa ma jedno przeznaczenie - zarządzanie dostępem do zasobu.

Szablon posiada 2 argumenty: \cname{T} -- Przechowywany typ zasobu, i \cname{M} -- typ mutexu.

Dostępny jest tylko konstruktor forwardujący - aby zrozumieć tę konstrukcję należy zapoznać się z pojęciami \note{idealnego przekazywania}{\link{https://eli.thegreenplace.net/2014/perfect-forwarding-and-universal-references-in-c}} i \note{tzw. uniwersalnych referencji}{\link{https://www.youtube.com/watch?v=wQxj20X-tIU}} - w skrócie polega to na tym że nasza klasa \tname{Resource} jest wrapperem przezroczystym dla klasy przechowywanej. Możemy prosto zdefiniować \inlineCpp{Resource<vector<int> > v \{1,2,3,4,5\}} tak jakby wrapper \cname{Resource} nie istniał.

Konstruktory kopiujące i przypisania są jawnie usunięte z 3 powodów:
\begin{itemize}
\item Zasób to z definicji coś niereplikowalnego i zajmującego unikalne miejsce w przestrzeni (rzeczywistej, czy pamięci komputera). Umożliwiając kopiowanie obiektów klasy \tname{Resource} złamałbym oczekiwania definiowane przez nazwę.
\item Z przyczyn praktycznych -- kopiowanie obiektu muteksu jest niemożliwe, a nawet gdyby dla innych mutexów było - to generowało by liczne problemy związane ze spójnością stanu muteksu jak i przechowywanego obiektu.
\item Prostota -- jak zobaczymy w przykładach przechowywanie jednego odwołania co wymusza na użytkowniku zaniechania tworzenia pajęczej sieci powiązań.
\end{itemize}

Dostęp do obiektu jest realizowany przez akcesor \fname{\tname{critical\_section}}. Nazwa jest starannie dobrana by była odpowiednio długa, treściwa, groźna i odróżniała się od reszty metod. Dostęp odbywa się za pomocą podanego w argumencie wizytatora który jest wywoływany w postaci sekcji krytycznej - muteks jest automatycznie zajmowany przez \tname{lock\_guard}, tuż przed wejściem wywołaniem argumentu i zwalniany tuż po powrocie. Użycie lock guard gwarantuje zwolnienie także w przypadku wyrzucenia wyjątku. Dodatkową zaletą użycia \tname{lock\_guard} jest możliwość zdefiniowania własnej implementacji dla customowych mutexów, a tym samym brak wymagań stawianych na klasę muteksu.

Powodem napisania tego prostego szablonu była potrzeba stworzenia interface'u biblioteki wywoływanego w trybie asynchronicznym, gdy każde z wywołań mogło odwoływać się do tego samego zestawu zasobów z czego każde wywołanie mogło operować na więcej niż jednym zasobie. Dzięki powyższemu szablonowi znacznie zmniejszyłem ryzyko zakleszczeń i skróciłem rozwlekłe sekcje krytyczne.
